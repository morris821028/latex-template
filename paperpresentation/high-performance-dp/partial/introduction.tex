\section{Introduction}

\subsection{DP Algorithms}
\begin{frame}
    \frametitle{Traditional Loop-based DP Algorithms}
	\begin{itemize}
		\item They are straightforward to implement, 
			sometimes have good spatial locality, and benefit from hardware
			prefetchers.
		\item Looping codes suffer in performance due to poor temporal cache
			locality.
	\end{itemize}
\end{frame}

\begin{frame}
    \frametitle{Flexible/Inflexible Kernels}
	\begin{itemize}
		\item Inflexible: The loops and the data in DP table cannot be suitably
			reordered in oreder to optimize for better spatial locality,
			parallelizations and/or vectorization.
		\item Flexible: For example, $i-j-k$ ordering run in $\mathcal{O}(n^3)$.
			The worst cases incurs $\Theta(n^3)$ cache misses in ideal cache model.
			$i-k-j$ ordering will incur only $\mathcal{O}(n^3/B + n^2)$ cache misses,
			where $B$ is the cache line size.
	\end{itemize}
\end{frame}

\subsection{CORDAC Technique}
\begin{frame}
    \frametitle{Cache-Oblivious Recursive Divide-and-Conquer Technique}
	\begin{itemize}
		\item Recursive algorithms are known to achieve excellent temporal locality.
		\item Several DP problems the recursive decomposition reduces the original 
			inflexible looping code into recursive functions and iterative kernels 
			that are predominantly flexible.
	\end{itemize}
\end{frame}

\begin{frame}
    \frametitle{Tiled Loops}
	\begin{itemize}
		\item Tiled Loops also achieve optimal cache performance.
		\item It does not improve its asymptotic parallelism.
	\end{itemize}
\end{frame}

\subsection{Contributions}
\begin{frame}
    \frametitle{Four DP problems}
	\begin{itemize}
		\item Parenthesization problem
		\item Gap problem
		\item Floyd-Warshall's all-pairs shortest path
		\item Protein accordion folding problem
	\end{itemize}
\end{frame}

\begin{frame}
    \frametitle{Marjor Contributions}
	\begin{itemize}
		\item Reduction to Flexible Computations for Better Parallelism
			and Optimizations
		\item Novel CORDAC Algorithms
		\item Optimizations and Experimental Analyses on Shared Memory Machines
		\item Comparison with Codes Generated by Polyhedral Compilers
		\item Energy, Power and Runtime Tradeoff
		\item Extension to Heterogeneous Platforms
	\end{itemize}
\end{frame}
