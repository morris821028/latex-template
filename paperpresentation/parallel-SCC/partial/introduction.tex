\section{Introduction}

\subsection{Strongly Connected Component}
\begin{frame}
    \frametitle{Strongly Connected Component}
	\begin{itemize}
		\item In the mathematical theory of directed graphs, a graph 
			is said to be strongly connected if every vertex is 
			\textbf{reachable} from every other vertex.
			\footnote{From Wikipedia, Strongly connected component}
	\end{itemize}
\end{frame}

\subsection{Algorithm}
\begin{frame}
	\frametitle{Serial Algorithms}
	\begin{itemize}
		\setlength\itemsep{1em}
		\item Tarjan's algorithm and Kosaraju's algorithm rely on 
			the hard-to-parallelize depth-first search (DFS). 
		\item These serial algorithms perform linear $O(n + m)$
			work in the random-access machine model of computation,
			where $n$ is \#vertices and $m$ is \#edges.
	\end{itemize}
\end{frame}

\begin{frame}
	\frametitle{Parallel Computing Strategy}
	\begin{itemize}
		\setlength\itemsep{1em}
		\item SCC decomposition is a useful preprocessing and data 
			reduction when analyzing large web graphs and networks
			constructed from online social network data.
	\end{itemize}
\end{frame}

\begin{frame}
	\frametitle{Parallel Algorithms}
	\begin{itemize}
		\setlength\itemsep{1em}
		\item Forward-Backward (FW-BW) Algorithm
		\item Coloring Algorithm
		\item Other Parallel SCC Approaches: There has been other 
			recent work aimed at improving FW-BW and coloring.
	\end{itemize}
\end{frame}

\subsection{Contributions}
\begin{frame}
	\frametitle{Multistep Method}
	\begin{itemize}
		\setlength\itemsep{1em}
		\item It is designed for SCC detection in large real-world graphs.
		\item For low-diameter networks, the single-threaded Multistep 
			approach is faster than the serial Tarjan's algorithm.
		\item Faster and exhibits better scaling than our
			implementations of the FW-BW and coloring algorithms.
		\item It is $8.9\times$ faster than the state-of-the-art Hong 
			et al. method on ItWeb.
	\end{itemize}
\end{frame}

\begin{frame}
	\frametitle{Modifed Multistep Method}
	\begin{itemize}
		\setlength\itemsep{1em}
		\item Multistep modified for CC is consistently faster than the
			coloring-based algorithm implementation in Ligra.
		\item Our novel BiCC algorithm demonstrates up to an $8\times$ 
			parallel speedup over a serial DFS-based approach.
		\item Our modified atomic-free and lock-free BFS averages a
			traversal rate of $1.4$ GTEPS (Giga traversed edges per
			second) over all tested networks.
	\end{itemize}
\end{frame}