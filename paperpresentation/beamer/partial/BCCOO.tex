\section{Matrix Data Format}

\subsection{COO Format}
\begin{frame}
    \frametitle{Common Coordinate Format}
	\begin{align*}
		A = \begin{bmatrix}
			0 & 0 & a & 0 & 0 & 0 & b & c \\
			0 & 0 & d & e & 0 & 0 & f & 0 \\
			0 & 0 & 0 & 0 & g & h & i & j \\
			k & l & 0 & 0 & m & n & o & p \\
		\end{bmatrix}
	\end{align*}
	\begin{align*}
		\text{Row index} &= [0, 0, 0, 1, 1, 1, 2, 2, 2, 2, 3, 3, 3, 3, 3, 3] \\
		\text{Col index} &= [2, 6, 7, 2, 3, 6, 4, 5, 6, 7, 0, 1, 4, 5, 6, 7] \\
		\text{Value}     &= [a, b, c, d, e, f, g, h, i, j, k, l, m, n, o, p]
	\end{align*}
    \begin{itemize} 
    	\item
    		Common Corrdinate Format (COO Format) has explicit storage for the
    		column and row indices.
    \end{itemize}	
\end{frame}

\subsection{BCOO Format}
\begin{frame}
	\frametitle{Block Common Coorindate Format (BCOO Format)}
	\begin{itemize}
		\item Block Common Coorindate Format extends the COO Format.
		\item First, using block size of $2 \times 2$ blocked COO format has 
			shown in figure.
	\end{itemize}
	\tikzset{%
  highlight/.style={rectangle,rounded corners,fill=red!15,draw,fill opacity=0.2,thick,inner sep=0pt}
}
\newcommand{\tikzmark}[2]{\tikz[overlay,remember picture,baseline=(#1.base)] \node (#1) {#2};}
%
\newcommand{\Highlight}[1][submatrix]{%
    \tikz[overlay,remember picture]{
      \node[highlight,fit=(left1.north west) (right1.south east)] (#1) {};
      \node[highlight,fit=(left2.north west) (right2.south east)] (#1) {};
    }
}


\begin{align*}
  A = \begin{bmatrix}
      0 & 0 & \tikzmark{left1}{a} & 0 & 0 & 0 & \tikzmark{left2}{b} & c \\
      0 & 0 & d & \tikzmark{right1}{e} & 0 & 0 & f & \tikzmark{right2}{0} \\
      0 & 0 & 0 & 0 & g & h & i & j \\
      k & l & 0 & 0 & m & n & o & p \\
    \end{bmatrix}
    \Highlight[first]
\end{align*}



\end{frame}

\subsection{BCCOO Format}
\begin{frame}
	\frametitle{From BCOO Format to BCCOO Format}
		Block Compressed Common Coordinate Format
		\begin{itemize}
			\item It uses difference function to compress row index array, and
				we get row index compression ratio equal $1/32$.
			\item If a difference value larger than 1, replace it with multiple 1s.
		\end{itemize}
\end{frame}

\begin{frame}
	\frametitle{BCCOO Format Example}
	\tikzset{%
  highlight/.style={rectangle,rounded corners,fill=red!15,draw,fill opacity=0.2,thick,inner sep=0pt}
}
\newcommand{\tikzmark}[2]{\tikz[overlay,remember picture,baseline=(#1.base)] \node (#1) {#2};}
%
\newcommand{\Highlight}[1][submatrix]{%
    \tikz[overlay,remember picture]{
      \node[highlight,fit=(left1.north west) (right1.south east)] (#1) {};
      \node[highlight,fit=(left2.north west) (right2.south east)] (#1) {};
    }
}


\begin{align*}
  A = \begin{bmatrix}
      0 & 0 & \tikzmark{left1}{a} & 0 & 0 & 0 & \tikzmark{left2}{b} & c \\
      0 & 0 & d & \tikzmark{right1}{e} & 0 & 0 & f & \tikzmark{right2}{0} \\
      0 & 0 & 0 & 0 & g & h & i & j \\
      k & l & 0 & 0 & m & n & o & p \\
    \end{bmatrix}
    \Highlight[first]
\end{align*}



\begin{align*}
	\text{Value} &= \left \{ \begin{bmatrix}
	a & 0\\ 
	d & e
	\end{bmatrix}, 
	\begin{bmatrix}
	b & c\\ 
	f & 0
	\end{bmatrix}, 
	\begin{bmatrix}
	0 & 0\\ 
	k & l
	\end{bmatrix}, 
	\begin{bmatrix}
	g & h\\ 
	m & n
	\end{bmatrix}, 
	\begin{bmatrix}
	i & j\\ 
	o & p
	\end{bmatrix}
	\right \} \\
	\text{Blocked Row index} &= [0, 0, 1, 1, 1] \\
	\text{Blocked Col index} &= [1, 3, 0, 2, 3] \\
	\text{Difference value}  &= [0, 1, 0, 0, 1] \\
	\text{Bit Flag}			 &= [1, 0, 1, 1, 0]	\\
\end{align*}

\end{frame}

\subsection{BCCOO+ Format}
\begin{frame}
	\frametitle{BCCOO+ Format}
	\begin{itemize}
		\item Reduce data transmission required for column index arrays
			using difference functions.
		\item Apply a segmented difference function on a column index array
			with each segment.
		\item The difference array is stored using the short data type 
			instead of the regular integer type.
		\item It is necessary to use a temporary buffer to store 
			intermediate results.
	\end{itemize}
\end{frame}

\begin{frame}
	\frametitle{BCCOO+ Format Example}
	\tikzset{%
  highlight/.style={rectangle,rounded corners,fill=red!15,draw,fill opacity=0.2,thick,inner sep=0pt}
}
\newcommand{\tikzmark}[2]{\tikz[overlay,remember picture,baseline=(#1.base)] \node (#1) {#2};}
%
\newcommand{\Highlight}[1][submatrix]{%
    \tikz[overlay,remember picture]{
      \node[highlight,fit=(left.north west) (right.south east)] (#1) {};
    }
}


\begin{align*}
  B = \begin{bmatrix}
      0 & 0 & a & 0 \\
      0 & 0 & d & e \\
      0 & 0 & 0 & 0 \\
      k & l & 0 & 0 \\
      \tikzmark{left}{0} & 0 & b & c \\
      0 & 0 & f & 0 \\
      g & h & i & j \\
      m & n & o & \tikzmark{right}{p}
    \end{bmatrix}
    \Highlight[Bsub1]
    ,
    \begin{bmatrix}
      \tikzmark{left}{0} & 0 & b & c \\
      0 & 0 & f & 0 \\
      g & h & i & j \\
      m & n & o & \tikzmark{right}{p}
    \end{bmatrix}
    \Highlight[Asub1]
\end{align*}

\end{frame}

\begin{frame}
	\frametitle{BCCOO+ Format Storage Example}
	\begin{align*}
		\text{Value} &= \left \{ \begin{bmatrix}
		a & 0\\ 
		d & e
		\end{bmatrix}, 
		\begin{bmatrix}
		b & c\\ 
		f & 0
		\end{bmatrix}, 
		\begin{bmatrix}
		0 & 0\\ 
		k & l
		\end{bmatrix}, 
		\begin{bmatrix}
		g & h\\ 
		m & n
		\end{bmatrix}, 
		\begin{bmatrix}
		i & j\\ 
		o & p
		\end{bmatrix}
		\right \} \\
		\text{Bit Flag}			 &= [0, 0, 0, 1, 0]	\\
		\text{Blocked Col index} &= [1, 0, 3, 2, 3] \qquad \text{(uncompressed)}\\
	\end{align*}
\end{frame}