\documentclass{beamer}
\usefonttheme[onlymath]{serif}

\usepackage{amsfonts}

% Code Block Setting
\usepackage{listings}
\lstset{language=C,
numberstyle=\footnotesize,
basicstyle=\ttfamily\footnotesize,
numbers=left,
stepnumber=1,
frame=shadowbox,
breaklines=true}

\usetheme{Warsaw}
% \usecolortheme{dove}

% Add frame number and total frame number in footline
\defbeamertemplate*{footline}{shadow theme}{%
    \leavevmode%
    \hbox{\begin{beamercolorbox}[wd=.5\paperwidth,ht=2.5ex,dp=1.125ex,leftskip=.3cm plus1fil,rightskip=.3cm]{author in head/foot}%
            \usebeamerfont{author in head/foot}\hfill\insertshortauthor
        \end{beamercolorbox}%
        \begin{beamercolorbox}[wd=.4\paperwidth,ht=2.5ex,dp=1.125ex,leftskip=.3cm,rightskip=.3cm plus1fil]{title in head/foot}%
            \usebeamerfont{title in head/foot}\insertshorttitle\hfill%
        \end{beamercolorbox}%
        \begin{beamercolorbox}[wd=.1\paperwidth,ht=2.5ex,dp=1.125ex,leftskip=.3cm,rightskip=.3cm plus1fil]{title in head/foot}%
            \hfill\insertframenumber\,/\,\inserttotalframenumber
    \end{beamercolorbox}}%
    \vskip0pt%
}

% Tikz related
\usepackage{tikz}
\usetikzlibrary{fit}
\usetikzlibrary{calc}
\usetikzlibrary{positioning}

% Number the figures
\setbeamertemplate{caption}[numbered]

% Add outline page at begining of each section
\AtBeginSection[]
{
    \begin{frame}<beamer>
        \frametitle{Outline}
        \tableofcontents[currentsection, hideallsubsections]
    \end{frame}
}

%%%%%%%%%%%%%%%%%%%%%%%%%%%%%%%%%%%%%%%%%%%%%

\title{Parallel Approaches to the String Matching Problem on the GPU}
\author{
    Saman Ashkiani\inst{1},\\
    Nina Amenta\inst{1},\\
    John D. Owens\inst{1}
}
\institute{
    \inst{1} University of California, Davis
}
\date{
    \tiny{SPAA'16 28th ACM Symposium on Parallelism in Algorithms and Architectures}\\
    \tiny{Presented by Shiang-Yun Yang}
}

\begin{document}
\begin{frame}
    \titlepage
\end{frame}

\section{Introduction}

\subsection{String Matching Algorithm}

\begin{frame}
	\frametitle{Some Algorithms}
	\begin{itemize}
		\setlength\itemsep{1em}
		\item Aho–Corasick string matching algorithm has asymptotic 
		worst-time complexity $O(n+m)$ in space $O(m)$.
		\item KMP algorithm,
		\item Boyer-Moore algorithm,
		\item Suffix tree,
		\item ... offline/online
	\end{itemize}
\end{frame}

\begin{frame}
    \frametitle{Rabin-Karp Algorithm}
	\begin{itemize}
		\setlength\itemsep{1em}
		\item Use hashing to find any one of a set of pattern 
			strings in a text.
		\item For text of length $n$ and $p$ patterns of combined 
		length $m$, its average and best case running time is $O(n+m)$
		 in space $O(p)$, but its worst-case time is $O(nm)$.
		\footnote{From Wikipedia, Rabin–Karp algorithm}
	\end{itemize}
\end{frame}


\begin{frame}
\lstinputlisting{code/rk.c}
\end{frame}

\subsection{Contributions}
\begin{frame}
	\frametitle{Contributions}
	\begin{itemize}
		\setlength\itemsep{1em}
		\item We propose three major RK-based binary matching algorithms:
			cooperative, divide-and-conquer, and a combination of 
			both (hybrid).
		\item We provide several optimizations: both in terms of the
			general algorithm and computations, as well as some
			hardware specific ones.
		\item We extend our RK-based algorithms to support characters
			from any general alphabets.
	\end{itemize}
\end{frame}
\section{Preliminaries}

\subsection{Basic Scenario}
\begin{frame}
	\frametitle{Basic Scenario}
	\begin{itemize}
		\setlength\itemsep{1em}
		\item Binary pattern of length $m$: 
			$X = x_0 \; \cdots \; x_{m-1}$
		\item Binary text of length $n$ ($n \ge m$): 
			$Y = y_0 \; \cdots \; y_{n-1}$
		\item Substring of $Y$: $Y[r] = y_r y_{r+1} \cdots y_{r+m-1}$
		\item Our Objective is to find all indices $r$ such that 
			$Y[r] = X$ for $0 \le r < n-m+1$.
	\end{itemize}
\end{frame}

\subsection{Graphic Processing Units}
\begin{frame}
	\frametitle{Graphic Processing Units}
	\begin{itemize}
		\item Both a computational hierarchy and a memory hierarchy.
		\item CUDA programming model
	\end{itemize}
\end{frame}

\subsection{Serial Rabin-Karp}
\begin{frame}
	\frametitle{Definition}
	For any $X \in \{0, 1\}^m$ :
	\begin{itemize}
		\setlength\itemsep{1em}
		\item $F(X) = 2^{m-1} x_0 + \cdots + 2 x_{m-2} + x_{m-1}$
		\item Pick a random prime number $p \in (1, mn^2)$
				$$F_{p}(X) \overset{p}{\equiv} F(X)$$
		\item Define two matrices,
			\begin{align}
				K(0) = \begin{bmatrix}
					1 & 0 \\ 
					1 & 1
					\end{bmatrix}, \;
				K(1) = \begin{bmatrix}
				1 & 1 \\ 
				0 & 1
				\end{bmatrix}
			\end{align}
		\item The fingerprint of $X$ is 
			$K(X) = K(x_0) \cdots K(x_{m-1})$.
	\end{itemize}
\end{frame}

\begin{frame}
	\frametitle{The RK Algorithm}
	\begin{description}
		\setlength\itemsep{1em}
		\item[(1)] Choose a random prime number $p \in (1, mn^2)$.
		\item[(2)] Compute fingerprints $F_p(X)$ and $F_p(Y[r])$ for 
			$r \in [0, n-m+1)$
		\item[(3)] Compare: if $F_p(Y[r])$ and $F_p(X)$ are equal,
			then $X$ and $Y[r]$ are equal with high probability, 
			otherwise there is no match.
	\end{description}
\end{frame}

\begin{frame}
	\frametitle{Compute Efficiently in Step 2}
	\begin{itemize}
		\item $F_p(Y[r+1])$ can be computed by using $F_p(Y[r])$:
			\begin{align}
			F_p(Y[r+1]) \overset{p}{\equiv} 2 (F_p(Y[r]) - 2^{m-1} y_r) + y_{r+m+1}
			\end{align}
		\item Compute $K_p(Y[r+1])$ based on $K_p(Y[r])$:
			\begin{align}
			K_p(Y[r+1]) \overset{p}{\equiv} A_p(y_r) K_p(Y[r]) K_p(y_{r+m+1})
			\end{align},
			where $A_p(x)$ is defined as the left inverse of $K_p(x)$ for any binary value $x \in \{0, 1\}$.
	\end{itemize}
\end{frame}



\section{Strategy}

\subsection{Cooperative Rabin-Karp}
\begin{frame}
	\frametitle{Definition}
	\begin{itemize}
		\item We define $\mathcal{S}$ and $\mathcal{T}$ vectors 
		as follow:
	\end{itemize}
	\begin{figure}
		\includegraphics[scale=0.50]{figure/fig-ST.png}
	\end{figure}
	\begin{itemize}
		\item With $p$ processors, a scan can be computed in 
			$O(n/p + \log p)$ time steps.
	\end{itemize}
\end{frame}


\begin{frame}
	\frametitle{Final Result}
	\begin{figure}
		\includegraphics[scale=0.50]{figure/fig-Kp.png}
	\end{figure}
	\begin{itemize}
		\item Scan operation can be computed cooperatively by a set of 
		independent threads or processors.
	\end{itemize}
\end{frame}

\subsection{Divide-and-Conquer Rabin-Karp}
\begin{frame}
	\frametitle{Divide and Conquer}
	\begin{itemize}
		\setlength\itemsep{1em}
		\item These parallel scans require intermediate communication
		 between different processors and cores.
		\item Assign different parts of the text of different 
		processors and process each part seperately. The final result
		is simply a union of matching results for each subproblem.
	\end{itemize}
\end{frame}

\begin{frame}
	Let $L$ denote the number of subtexts. Then, if we
	show each subtext as $Y^l$, for $0 \le l < L$, the division
	process can be shown as:
	\begin{figure}
		\includegraphics[scale=0.50]{figure/fig-DC.png}
	\end{figure}
	where each subtext has $g = (n-m+1)/L$ \textit{exclusive} 
	characters, plus $m-1$ \textit{overlapped} characters.
	\begin{itemize}
		\item If $L = 1$, DRK will be identical to CRK.
	\end{itemize}
\end{frame}

\subsection{Hybrid Rabin-Karp}
\begin{frame}
	\begin{itemize}
		\setlength\itemsep{1em}
		\item We define \textit{Hybrid RK} (HRK) as a method in which 
		the main text is divied into subtexts (Eq. (7)) and then each 
		subtext is assigned to a group of processors.
		\item If $L = 1$, HRK will be identical to CRK.
	\end{itemize}
\end{frame}

\begin{frame}
	\begin{figure}
		\includegraphics[scale=0.50]{figure/fig-all.png}
	\end{figure}
\end{frame}

\subsection{Theoretical Analysis}
\begin{frame}
	\frametitle{Theoretical Analysis}
	\begin{itemize}
		\setlength\itemsep{1em}
		\item The finite number of $p$ processors.
		\item Assume that all processors have access to a globally 
		shared memory.
	\end{itemize}
	\begin{figure}
		\includegraphics[scale=0.40]{figure/fig-complexity.png}
	\end{figure}
\end{frame}

\begin{frame}
	\frametitle{Theoretical Conclusions}
	\begin{itemize}
		\item We expect to see an approximate linear increase in the running time of all our parallel alternatives.
		\item CRK is superior to the other algorithms, and for all
		larger patterns.
	\end{itemize}
	\begin{figure}
		\includegraphics[scale=0.25]{figure/fig-result.png}
	\end{figure}
\end{frame}
\section{Implementation}

\subsection{Cooperative RK}
\begin{frame}
	\frametitle{Cooperative RK}
	\begin{itemize}
		\setlength\itemsep{1em}
		\item Use the scan operation provided by the Thrust library.
		\item All intermediate results are inthe form of arrays 2-by-2 
		integer matrices: $\mathcal{L}$, $\mathcal{A}$, $\mathcal{S}$,
		$\mathcal{T}$ each requires 16B per input character.
	\end{itemize}
\end{frame}

\subsection{Divide-and-Conquer RK}
\begin{frame}
	\frametitle{Divide-and-Conquer RK}
	\begin{itemize}
		\setlength\itemsep{1em}
		\item For binary patterns of size less than or equal 32 
		characters, we can use 32-bit registers.
		\item Using \texttt{SHIFT} and \texttt{AND} to replace module operators.
		(DRK without module operator: DRK-WOM)
		\item We read elements as \texttt{char4} vectors to better
		exploit the available memory bandwidth.
	\end{itemize}
\end{frame}

\section{Expanding}

\subsection{General alphabets}
\begin{frame}
	\frametitle{General alphabets}
	\begin{itemize}
		\item We have an alphabet set $\Sigma$  such that each character $a \in \Sigma$ requires at most 
		$\sigma = \lceil \log \Sigma \rceil$ bits of information.
	\end{itemize}
	\begin{figure}
		\includegraphics[scale=0.50]{figure/fig-general-alpha.png}
	\end{figure}
\end{frame}

\subsection{Two-stage matching}
\begin{frame}
	\frametitle{Two-stage matching}
	\begin{figure}
		\includegraphics[scale=0.40]{figure/fig-2-stage.png}
	\end{figure}
\end{frame}
\section{Real-world Scenarios}

\subsection{Real-world Scenarios}
\begin{frame}
	\frametitle{Real-world Scenarios}
\end{frame}

\subsection{False Positives}
\begin{frame}
	\frametitle{False Positives}
\end{frame}




\end{document}
