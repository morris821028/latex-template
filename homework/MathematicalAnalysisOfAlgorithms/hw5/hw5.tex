%=======================02-713 LaTeX template, following the 15-210 template==================
%
% You don't need to use LaTeX or this template, but you must turn your homework in as
% a typeset PDF somehow.
%
% How to use:
%    1. Update your information in section "A" below
%    2. Write your answers in section "B" below. Precede answers for all 
%       parts of a question with the command "\question{n}{desc}" where n is
%       the question number and "desc" is a short, one-line description of 
%       the problem. There is no need to restate the problem.
%    3. If a question has multiple parts, precede the answer to part x with the
%       command "\part{x}".
%    4. If a problem asks you to design an algorithm, use the commands
%       \algorithm, \correctness, \runtime to precede your discussion of the 
%       description of the algorithm, its correctness, and its running time, respectively.
%    5. You can include graphics by using the command \includegraphics{FILENAME}
%
\documentclass[11pt]{article}

\usepackage[UTF8, heading = false, scheme = plain]{ctex} % chinese !!!!
\usepackage{amsmath,amssymb,amsthm}
\usepackage{graphicx}
\usepackage[margin=1in]{geometry}
\usepackage{fancyhdr}
\usepackage{CJK}
\usepackage{manfnt}
\usepackage{listings}

\usepackage{algorithm}% http://ctan.org/pkg/algorithm
\usepackage{algpseudocode}% http://ctan.org/pkg/algorithmicx


\setCJKmainfont{STHeiti} % chinese type

\setlength{\parindent}{0pt}
\setlength{\parskip}{5pt plus 1pt}
\setlength{\headheight}{13.6pt}

% define
\newcommand\question[2]{\vspace{.25in}\hrule\textbf{#1: #2}\vspace{.5em}\hrule\vspace{.10in}}
\renewcommand\part[1]{\vspace{.10in}\textbf{#1}}
\newcommand\correctness{\vspace{.10in}\textbf{Correctness: }}
\newcommand\runtime{\vspace{.10in}\textbf{Running time: }}
\pagestyle{fancyplain}

% header
\lhead{\textbf{\NAME\ (\ANDREWID)}}
\chead{\textbf{HW\HWNUM}}
\rhead{演算法數學分析, \today}


% start content
\begin{document}\raggedright
%Section A==============Change the values below to match your information==================
\newcommand\NAME{Shiang-Yun Yang 楊翔雲}  % your name
\newcommand\ANDREWID{R04922067}     % your andrew id
\newcommand\HWNUM{5}              % the homework number
%Section B==============Put your answers to the questions below here=======================

% no need to restate the problem --- the graders know which problem is which,
% but replacing "The First Problem" with a short phrase will help you remember
% which problem this is when you read over your homeworks to study.


\question{Problem 1} {5-14 Prove identity (5.25) by negating the upper index in
	Vandermonde's convolution (5.22). Then show that another negation yields (5.26).
}

\begin{flalign*}
& \sum_k \binom{r}{m+k} \binom{s}{n-k} = \binom{r+s}{m+n}	
		&& \text{, integers } m, n && \text{(5.22)} \\
& \sum_{k \le l} \binom{l-k}{m} \binom{s}{k-n} (-1)^k = (-1)^{l+m} \binom{s-m-1}{l-m-n}
		&& \text{, integers } l, m, n \ge 0 && \text{(5.25)} \\
& \sum_{0 \le k \le l} \binom{l-k}{m} \binom{q+k}{n} = \binom{l+q+1}{m+n+1}
		&& \text{, integers } l, m > 0 \text{, integers } n \ge q \ge 0 && \text{(5.26)}
\end{flalign*}

\part{Answer:}

\begin{itemize}
	\item Part 1
	\begin{align*}
		& \sum_{k \le l} \binom{l-k}{m} \binom{s}{k-n} (-1)^k \\
		& = \sum_{k \le l} \binom{l-k}{l-k-m} \binom{s}{k-n} (-1)^k
		&& \because \binom{l-k}{m} = \binom{l-k}{l-k-m} \\
		& = \sum_{k \le l} (-1)^{l-k-m} \binom{-m-1}{l-k-m} \binom{s}{k-m} (-1)^k
		&& \because \binom{r}{k} = (-1)^k \binom{k-r-1}{k} \\
		& = \sum_{k \le l} (-1)^{l-m} \binom{-m-1}{(l-m)-k} \binom{s}{-m+k}
		&& \because \text{(5.22), } \sum_k \binom{r}{m+k} \binom{s}{n-k} = \binom{r+s}{m+n} \\
		& = (-1)^{l-m} \binom{s-m-1}{l-m+n}
	\end{align*}
	\item Part 2
	\begin{align*}
		& \sum_{0 \le k \le l} \binom{l-k}{m} \binom{q+k}{n} \\
		&= \sum_{0 \le k \le l} \binom{l-k}{m} \binom{q+k}{q+k-n} \\
		&= \sum_{0 \le k \le l} \binom{l-k}{m} (-1)^{q+k-n}\binom{-n-1}{q+k-n} \\
		&= (-1)^{q-n} \sum_{0 \le k \le l} \binom{l-k}{m} \binom{-n-1}{q+k-n} (-1)^k \\
		&= (-1)^{q-n} (-1)^{l+m} \binom{-n-1-m-1}{l-m-n-q} \\
		&= (-1)^{q-n} (-1)^{l+m} (-1)^{l-m-n+q} \binom{l+q+1}{l-m-n+q} \\
		&= (-1)^{2l + 2n + 2q} \binom{l+q+1}{l+q-n-m} \\
		&= \binom{l+q+1}{n+m+1}
	\end{align*}
\end{itemize}

\question{Problem 2} {5-43}

\part{Answer:}

\question{Problem 3} {5-58}

\part{Answer:}

\question{Problem 4} {5-24}

\part{Answer:}

\question{Problem 5} {Evaluate $$\frac{1}{N} \sum_{1 \le k \le N} \sum_{t} \frac{t \binom{N-k}{t} \binom{k-1}{t}}{\binom{N-1}{k-1}}$$}

\part{Answer:}

\begin{flalign*}
\frac{1}{N} \sum_{1 \le k \le N} \sum_{t} \frac{t \binom{N-k}{t} \binom{k-1}{t}}{\binom{N-1}{k-1}} 
	&= \frac{1}{N} \sum_{1 \le k \le N} \frac{1}{\binom{N-1}{k-1}} \sum_{t} t \binom{N-k}{t} \binom{k-1}{t} \\
	&= \frac{1}{N} \sum_{1 \le k \le N} \frac{1}{\binom{N-1}{k-1}} \sum_{t} \binom{N-k}{t} \binom{k-2}{t-1} (k-1) 
		&& \because \binom{r}{k} = \frac{r}{k} \binom{r-1}{k-1} \\
	&= \frac{1}{N} \sum_{1 \le k \le N} \frac{1}{\binom{N-1}{k-1}} (k-1) \binom{N-2}{N-k-1}
		&& \because \text{(5.23)} \\
	&= \frac{1}{N} \sum_{1 \le k \le N} \frac{1}{\binom{N-1}{k-1}} (k-1) \binom{N-2}{k-1}
		\\
	&= \frac{1}{N} \sum_{1 \le k \le N} (k-1) \frac{N-k}{N-1} \\
	&= \frac{1}{N(N-1)} \sum_{1 \le k \le N} \left [ N(k-1) + (k-k^2) \right ] \\
	&= \frac{1}{N(N-1)} \left [ N \frac{N(N-1)}{2} + \frac{N(N+1)}{2} - \frac{N(N+1)(2N+1)}{6} \right ] \\
	&= \frac{N}{2} + \frac{N+1}{2(N-1)} - \frac{(N+1)(2N+1)}{2(N-1)} 
		= \frac{N}{2} - \frac{N(N+1)}{N-1}\\
\end{flalign*}

\question{Problem 6} {
}

\part{Answer:}

\end{document}