%=======================02-713 LaTeX template, following the 15-210 template==================
%
% You don't need to use LaTeX or this template, but you must turn your homework in as
% a typeset PDF somehow.
%
% How to use:
%    1. Update your information in section "A" below
%    2. Write your answers in section "B" below. Precede answers for all 
%       parts of a question with the command "\question{n}{desc}" where n is
%       the question number and "desc" is a short, one-line description of 
%       the problem. There is no need to restate the problem.
%    3. If a question has multiple parts, precede the answer to part x with the
%       command "\part{x}".
%    4. If a problem asks you to design an algorithm, use the commands
%       \algorithm, \correctness, \runtime to precede your discussion of the 
%       description of the algorithm, its correctness, and its running time, respectively.
%    5. You can include graphics by using the command \includegraphics{FILENAME}
%
\documentclass[11pt,fleqn]{article}

\usepackage[UTF8, heading = false, scheme = plain]{ctex} % chinese !!!!
\usepackage{amsmath,amssymb,amsthm}
\usepackage{mathtools}
\usepackage{graphicx}
\usepackage[margin=1in]{geometry}
\usepackage{fancyhdr}
\usepackage{CJK}
\usepackage{manfnt}
\usepackage{listings}
\usepackage{epsdice}

\usepackage{algorithm}% http://ctan.org/pkg/algorithm
\usepackage{algpseudocode}% http://ctan.org/pkg/algorithmicx


\setCJKmainfont{STHeiti} % chinese type

\setlength{\parindent}{0pt}
\setlength{\parskip}{5pt plus 1pt}
\setlength{\headheight}{13.6pt}

% define
\newcommand\question[2]{\vspace{.25in}\hrule\textbf{#1: #2}\vspace{.5em}\hrule\vspace{.10in}}
\renewcommand\part[1]{\vspace{.10in}\textbf{#1}}
\newcommand\correctness{\vspace{.10in}\textbf{Correctness: }}
\newcommand\runtime{\vspace{.10in}\textbf{Running time: }}
\pagestyle{fancyplain}

% header
\lhead{\textbf{\NAME\ (\ANDREWID)}}
\chead{\textbf{HW\HWNUM}}
\rhead{演算法數學分析, \today}


% start content
\begin{document}\raggedright
%Section A==============Change the values below to match your information==================
\newcommand\NAME{Shiang-Yun Yang 楊翔雲}  % your name
\newcommand\ANDREWID{R04922067}     % your andrew id
\newcommand\HWNUM{9}              % the homework number
%Section B==============Put your answers to the questions below here=======================

% no need to restate the problem --- the graders know which problem is which,
% but replacing "The First Problem" with a short phrase will help you remember
% which problem this is when you read over your homeworks to study.

\begin{itemize}
	\item Euler's Summation Formula 
		\begin{align*}
			\sum_{a \le k < b} f(k) = \int^{b}_{a} f(x) dx + 
				\left.\begin{matrix} \sum\limits_{k=1}^{m} \dfrac{B_k}{k!} f^{(k-1)} (x) \end{matrix}\right|^{b}_{a} + R_m, && \text{(9.67)}\\
			\text{where } R_m = (-1)^{m+1} \in^{b}_{a} \dfrac{B_m(\{x\})}{m!} f^{(m)}(x) dx,
				\quad
				\begin{matrix}
					\text{integers} a \le b \\
					\text{integer} m \ge 1
				\end{matrix}
				&& \text{(9.68)}
		\end{align*}
	\item Bernoulli numbers, $B_0 = 1, \; B_1 = -\frac{1}{2}, \; B_2 = \frac{1}{6}, B_4 = -\frac{1}{30}$
\end{itemize}

\question{Problem 1} {9-7 Estimate $\sum_{k \ge 0} e^{-k/n}$ with absolute
	error $O(n^{-1})$.
}

\part{Answer:}

代入 Euler's Summation Formula 9.67 展開

\begin{align*}
	\int f(x) dx &= e^{x/n} \cdot (-n) \\
	f(k) &= e^{-k/n} \\
	f^{(1)}(k) &= e^{-k/n} \cdot (-1)^1(1/n) \\
	f^{(2)}(k) &= e^{-k/n} \cdot (-1)^2(1/n^2)
\end{align*}

\begin{align*}
\sum_{k \ge 0} e^{-k/n} 
	&= \int^{\infty}_{0} e^{-x/n} dx + B_1 \cdot f(x) 
			\left.\phantom{0\over0} \right|^{\infty}_{0} + 
		\frac{B_2}{2!} \cdot f'(x) 
			\left.\phantom{0\over0} \right|^{\infty}_{0} + \cdots
			\\
	&= n + B_1 + \frac{B_2}{2!} \cdot \frac{1}{n} + \cdots \\
	&= n + \frac{1}{2} + O(n^{-1}) && \blacksquare
\end{align*}

\newpage

\question{Problem 2} {9-32 Evaluate $e^{H_n + H^{(2)}_n}$ with absolute error $O(n^{-1})$.
}

\part{Answer:}

\begin{itemize}
	\item 從 Euler's Summation Formula 展開,得知 \footnote{http://math.stackexchange.com/questions/174345/compute-sum-frac1k2-using-euler-maclaurin-formula}	
		\begin{align*}
			H_n^{(2)} &= \sum_{k=1}^{n} \frac{1}{k^2} \\
			 	&\approx
					\pi^2/6 - \frac{1}{n} + \frac{1}{2n^2} - \frac{1}{6n^3} 
						+ \frac{1}{30n^5} - \frac{1}{42n^7} + \frac{1}{30n^9} 
							- \frac{5}{66n^{11}} \\
				&\approx
					\pi^2/6 - 1/n + O(n^{-2})
		\end{align*}
	\item 已知 $H_n = \ln n + \gamma + \frac{1}{2n} - \frac{1}{12n^2} + \frac{1}{120n^4} + O(\frac{1}{n^6})$,代回原式
		\begin{align*}
			e^{H_n + H^{(2)}_n} 
				&= \exp(H_n) \cdot \exp(H^{(2)}_n) \\
				&= \exp\left(\ln n + \gamma + \frac{1}{2n} + O(\frac{1}{n^2})\right)
					\cdot \exp\left(\pi^2/6 - 1/n + O(n^{-2})\right) \\
				&= n e^{\gamma + \pi^2/6} \cdot
					\left(1 + \frac{1}{2n} + O(n^{-2})\right) 
					\cdot \left(1 - \frac{1}{n} + O(n^{-2})\right)
					\cdot (1 + O(n^{-2})) \\
				&= n e^{\gamma + \pi^2/6} \left(1 - \frac{1}{2} n^{-1} + O(n^{-2})\right)
					&& \blacksquare
		\end{align*}
\end{itemize}

\question{Problem 3} {9-35 Evaluate $\sum_{k=1}^{n} 1/k H_k$ with absolut error $O(1)$.
}

\part{Answer:}

\begin{align*}
	\sum\limits_{k=1}^{n-1} \frac{1}{k H_k} 
		&\approx \sum\limits_{k=1}^{n-1} \frac{1}{k(\ln k + O(1))} 
		 \approx \sum\limits_{k=1}^{n-1} \frac{1}{k\ln k + O(k)} \\
		&\approx \sum\limits_{k=1}^{n-1} \frac{1}{k\ln k} 
			\left(1 + \frac{O(k)}{k \ln k}\right)^{-1} && \text{代入 (9.34)}\\
		&\approx \sum\limits_{k=1}^{n-1} \frac{1}{k\ln k} 
			\left(1 + O\left(\frac{1}{\ln k}\right)\right) 
		= \sum\limits_{k=1}^{n-1} \left(\frac{1}{k\ln k} 
			+ O\left(\frac{1}{k \ln^2 k}\right)\right) \\
		&\approx \sum\limits_{k=1}^{n-1} \frac{1}{k\ln k} \quad + O(1) \\
\end{align*}

根據 Euler's Summation Formula 求近似

\begin{align*}
	&\sum\limits_{k=1}^{n-1} \frac{1}{k\ln k} \quad + O(1) \\
	&\approx \int^{n}_{1} \frac{1}{x \ln x} dx + B_1 \frac{x}{x \ln x} 
		\left.\phantom{0\over0} \right|^{n}_{1} + \frac{B_2}{2!} 
		\frac{\ln x + 1}{x^2 \ln^2 x} \left.\phantom{0\over0} \right|^{n}_{1} + O(1) \\
	&= \ln \ln x \left.\phantom{0\over0} \right|^{n}_{1} 
		- \frac{1}{2} \frac{1}{x \ln x} \left.\phantom{0\over0} \right|^{n}_{1}
		+ \frac{1}{2} \frac{1}{6} \frac{\ln x + 1}{x^2 \ln^2 x} 
			\left.\phantom{0\over0} \right|^{n}_{1} + O(1) \\
	&= \ln \ln n + O(1) \phantom{{0\over0}\over0} && \blacksquare
\end{align*}

\question{Problem 4} {9-36 Evaluate $S_n = \sum\limits_{k=1}^{n} 1 / (n^2 + k^2)$ with
	absolute error $O(n^{-5})$.
}

\part{Answer:}

\begin{align*}
	S_n &= \sum\limits_{k=1}^{n} \frac{1}{n^2 + k^2} \\
		&= \sum_{0 \le k < n} \left(\frac{1}{n^2 + k^2}\right) - \frac{1}{n^2} + \frac{1}{2n^2} \\
		&= \int^{n}_{0} \frac{1}{n^2 + x^2} dx + 
			\frac{B_1}{1!} \frac{1}{n^2+x^2} \left.\phantom{0\over0} \right|^{n}_{0} + 
			\frac{B_2}{2!} \frac{-2x}{(n^2+x^2)^2} \left.\phantom{0\over0} \right|^{n}_{0}
			+ O(n^{-5}) - \frac{1}{n^2} + \frac{1}{2n^2} \\
		&= \frac{\tan^{-1}(n/x)}{n} \left.\phantom{0\over0} \right|^{n}_{0}
			- \frac{1}{2} \frac{1}{n^2 + x^2} \left.\phantom{0\over0} \right|^{n}_{0} 
			+ \frac{1}{2} \frac{1}{6} \frac{-2x}{(n^2+x^2)^2}
				\left.\phantom{0\over0} \right|^{n}_{0} + O(n^{-5}) 
				- \frac{1}{n^2} + \frac{1}{2n^2} \\
		&= \frac{\pi}{4} n^{-1} + \frac{1}{4} n^{-2} - \frac{1}{24} n^{-3} + O(n^{-5}) 
				- \frac{1}{n^2} + \frac{1}{2n^2}\\
		&= \frac{\pi}{4} n^{-1} - \frac{1}{4} n^{-2} - \frac{1}{24} n^{-3} + O(n^{-5}) 
			&& \blacksquare
\end{align*}

\end{document}