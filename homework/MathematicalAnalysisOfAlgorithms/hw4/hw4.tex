%=======================02-713 LaTeX template, following the 15-210 template==================
%
% You don't need to use LaTeX or this template, but you must turn your homework in as
% a typeset PDF somehow.
%
% How to use:
%    1. Update your information in section "A" below
%    2. Write your answers in section "B" below. Precede answers for all 
%       parts of a question with the command "\question{n}{desc}" where n is
%       the question number and "desc" is a short, one-line description of 
%       the problem. There is no need to restate the problem.
%    3. If a question has multiple parts, precede the answer to part x with the
%       command "\part{x}".
%    4. If a problem asks you to design an algorithm, use the commands
%       \algorithm, \correctness, \runtime to precede your discussion of the 
%       description of the algorithm, its correctness, and its running time, respectively.
%    5. You can include graphics by using the command \includegraphics{FILENAME}
%
\documentclass[11pt]{article}

\usepackage[UTF8, heading = false, scheme = plain]{ctex} % chinese !!!!
\usepackage{amsmath,amssymb,amsthm}
\usepackage{graphicx}
\usepackage[margin=1in]{geometry}
\usepackage{fancyhdr}
\usepackage{CJK}
\usepackage{manfnt}
\usepackage{listings}

\usepackage{algorithm}% http://ctan.org/pkg/algorithm
\usepackage{algpseudocode}% http://ctan.org/pkg/algorithmicx


\setCJKmainfont{STHeiti} % chinese type

\setlength{\parindent}{0pt}
\setlength{\parskip}{5pt plus 1pt}
\setlength{\headheight}{13.6pt}

% define
\newcommand\question[2]{\vspace{.25in}\hrule\textbf{#1: #2}\vspace{.5em}\hrule\vspace{.10in}}
\renewcommand\part[1]{\vspace{.10in}\textbf{#1}}
\newcommand\correctness{\vspace{.10in}\textbf{Correctness: }}
\newcommand\runtime{\vspace{.10in}\textbf{Running time: }}
\pagestyle{fancyplain}

% header
\lhead{\textbf{\NAME\ (\ANDREWID)}}
\chead{\textbf{HW\HWNUM}}
\rhead{演算法數學分析, \today}


% start content
\begin{document}\raggedright
%Section A==============Change the values below to match your information==================
\newcommand\NAME{Shiang-Yun Yang 楊翔雲}  % your name
\newcommand\ANDREWID{R04922067}     % your andrew id
\newcommand\HWNUM{4}              % the homework number
%Section B==============Put your answers to the questions below here=======================

% no need to restate the problem --- the graders know which problem is which,
% but replacing "The First Problem" with a short phrase will help you remember
% which problem this is when you read over your homeworks to study.


\question{Problem 1} {Prove that 
	$$\sum_{k|m} \frac{1}{k} \sum_{d|k} \mu(k/d)c^d 
		= \frac{1}{m} \sum_{d|m} \phi(d) c^{\frac{m}{d}}$$
}

\part{Answer:}

從右式開始推導

\begin{align*}
	\frac{1}{m} \sum_{d|m} \phi(d) c^{\frac{m}{d}} &= 
		\frac{1}{m} \sum_{d|m} \sum_{k|d} \mu(k) \frac{d}{k} c^{\frac{m}{d}}
			&& \text{性質} \phi(d) = \sum_{k|d} \mu(d) \frac{d}{k} \\
		&= \sum_{d|m} \sum_{k|d} \mu(k) \frac{d}{mk} c^{\frac{m}{d}} \\
		&= \sum_{d|m} \sum_{k|(m/d)} \mu(k) \frac{m/d}{mk} c^{\frac{m}{m/d}} 
			&& \text{根據對稱性質,重新排列 } d \\
		&= \sum_{d|m} \sum_{k|(m/d)} \mu(k) \frac{1}{kd} c^d \\
		&= \sum_{kd|m} \mu(k) \frac{1}{kd} c^d 
			&& d|m \text{ and } k|(m/d) \Rightarrow kd|m \\
		&= \sum_{a|m} \sum_{b|a} \mu(a/b) \frac{1}{a} c^b 
			&& \text{令 } kd=a, \quad d = b \text{ 改寫等式} \\
		&= \sum_{a|m} \frac{1}{a} \sum_{b|a} \mu(a/b) c^b 
			&& \text{移項} \\
		&= \sum_{k|m} \frac{1}{k} \sum_{d|k} \mu(k/d) c^d 
			&& \text{變數重新命名,得到左式} \\
\end{align*}

根據上述推導,從右式推導到左式,每一步均為雙向,故等式成立。

\question{Problem 2} {4-29 The text describes a correspondence between
	binary real numbers $x = (.b_1 b_2 b_3 \cdots)_2$ in $[0, 1)$ and 
	Stern-Brocot real numbers $\alpha = B_1 B_2 B_3 \cdots$ in 
	$[0, \infty)$. If $x$ corresponds to $\alpha$ and $x \neq 0$, 
	what number corresponds to $1-x$ ?
}

\part{Answer:}

\begin{itemize}
	\item 已知 $b_i \in \{0, 1\}$,$B_i \in \{\texttt{L}, \texttt{R}\}$。
	\item 當 $x = (.b_1 b_2 b_3 \cdots)_2$,則 $1-x = (.b'_1 b'_2 b'_3 \cdots)_2$,
		其中 $b'_i = 1 - b_i$。
	\item 而 $(.b'_1 b'_2 b'_3 \cdots)_2$ 對應的數值 Stern-Brocot real numbers 
		$\beta = B'_1 B'_2 B'_3 \cdots$,滿足
		\begin{align*}
			B'_i = \left\{\begin{matrix}
				\texttt{L} & \text{if } B_i = \texttt{R}, \; i > 1 \\
				\texttt{R} & \text{if } B_i = \texttt{L}, \; i > 1 \\
				\texttt{L} & i = 1
			\end{matrix}\right.
		\end{align*}
\end{itemize}

\question{Problem 3} {4-38 Prove that if $a \perp b$ and $a > b$ then
	\begin{align*}	
		\text{gcd}(a^m - b^m, a^n - b^n) = a^{\text{gcd}(m, n)}
			- b^{\text{gcd}(m, n)} && 0 \le m < n.
	\end{align*}
	(All variables are integers.) \textit{Hint} Use Euclid's algorithm.
}

\part{Answer:}

\begin{align*}
	\text{gcd}(a^m - b^m, a^n - b^n) 
		&= \text{gcd}(a^m - b^m, a^n - b^n - (a^m - b^m) a^{-m+n}) && \text{左式乘上 } a^{-m+n}\\
		&= \text{gcd}(a^m - b^m, a^n - b^n - a^n + b^m a^{-m+n}) \\
		&= \text{gcd}(a^m - b^m, b^n + b^m a^{-m+n}) \\
		&= \text{gcd}(a^m - b^m, b^m (a^{n-m} - b^{n-m})) && \text{共同提出} b^m\\
		&= \text{gcd}(a^m - b^m, a^{n-m} - b^{n-m}) && \text{gcd}(a^m - b^m, b^m) = 1\\
		&= \vdots \\
		&= \text{gcd}(a^{\text{gcd}(m, n)} - b^{\text{gcd}(m, n)}, a^{\text{gcd}(m, n)} - b^{\text{gcd}(m, n)}) \\
		&= a^{\text{gcd}(m, n)} - b^{\text{gcd}(m, n)}
\end{align*}

\question{Problem 4} {4-41
	\begin{description}
		\item[a] Show that if $p \mod 4 = 3$, there is no integer $n$ such
			that $p$ divides $n^2 + 1$. \textit{Hint}: Use Fermat's theorem.
		\item[b] But show that if $p \mod 4 = 1$, there is such an integer.
			\textit{Hint}: Write $(p-1)!$ as $(\prod_{k=1}^{(p-1)/2} k(p-k))$
			and think about Wilson's theorem.
	\end{description}
}

\part{Answer:}

\begin{description}
	\item[a] 假設存在一個整數 $n$ 符合 $n^2 + 1 \equiv 0 (\text{mod } p)$,且質數 
		$p$ 滿足 $p \mod 4 = 3$。
		\begin{align*}
			n^2 + 1 \equiv 0 (\text{mod } p) 
			&\Rightarrow n^2 \equiv -1 (\text{mod } p) \\
			&\Rightarrow (n^2)^{(p-1)/2} \equiv (-1)^{(p-1/2)} (\text{mod } p) \\
			&\Rightarrow n^{p-1} \equiv -1 (\text{mod } p) && (p-1)/2 \text{ 為奇數}
		\end{align*}
		根據費馬小定理 $a^{p-1} \equiv 1 \mod p, \; \text{if } \gcd(a, p) = 1$,上述等式不成立。
		矛盾得證,當質數 $p \mod 4 = 3$時,不存在任何整數 $n$ 滿足 $n^2+1 \equiv 0 \mod p$。
	\item[b] 從 Wilson's theorem 得知,當 $p$ 是質數 $(p-1)! \equiv -1 \mod p$。
		\begin{align*}
			(p-1)! 	&\equiv \prod_{k=1}^{(p-1)/2} k(p-k) \mod p && \text{重新排列} \\
					&\equiv \prod_{k=1}^{(p-1)/2} kp - k^2 \mod p & \\
					&\equiv \prod_{k=1}^{(p-1)/2} - k^2 \mod p && kp \equiv 0 \mod p\\
					&\equiv [ ((p-1)/2)! ]^2 \mod p && (p-1)/2 \text{ 為偶數} \\
					&\equiv -1 \mod p && \text{Wilson's theorem}
		\end{align*}
		得到對於任何質數 $p$ 滿足 $p \mod 4 = 1$,都存在 $n = ((p-1)/2)!$ 符合
		滿足 $n^2+1 \equiv 0 \mod p$。
\end{description}

\question{Problem 5} {To find the average running time, it suffices to find the average 
	value of $A_n$, the number of times the inner loop is executed. The average (or 
	expected) value of $A_n$ is $E(A_n) = B_N / n!$.
\begin{algorithm}
  \caption{Insertion Sort Algorithm}\label{insert}
  \begin{algorithmic}[1]
  	\For{ $i := 2$ \texttt{to} $n$}			\Comment{$n$}
		\State \textit{value} $:= X[i]$		\Comment{$n-1$}
		\State \textit{place} $:= i-1$		\Comment{$n-1$}
		\While{$\textit{place} \ge 1$ \texttt{and} $X[\textit{place}] \ge \textit{value}$}		\Comment{$A_n + n - 1$}
			\State $X[\textit{place}+1] := X[\textit{place}]$	\Comment{$A_n$}
			\State \textit{place} $:= \textit{place} - 1$	\Comment{$A_n$}
		\EndWhile
		\State $X[\textit{place}+1] := \textit{value}$	\Comment{$n-1$}
	\EndFor
  \end{algorithmic}
\end{algorithm}
	\begin{description}
		\item[(a)] Find a simple expression for $E(A_n)$ in terms of $n$. Do that
			by deriving a recurrence relation for $B_n$ in terms of $B_{n-1}$, then
			solving it by using the appropriate "summation factor." Compute the
			average running time $4(E(A_n)) + 6n - 5$.
		\item[(b)] One way to speed up the algorithm is to get rid of the 
			"place $\ge$ 1" expression, since it is seldom needed. To do that, 
			we must initialize $X[0]$ to $-\infty$. Now the "\texttt{while}" statement 
			takes only one time unit to execute. Does this alteration affect the value
			of $A_n$ ? Why or why not ? What is the resulting average running time
			of the algorithm ?
		\item[(c)] Is this algorithm \textit{stable} ? If not, can you modify the
			algorithm to make it stable ? A sorting method \textit{stable} if equal
			keys remain in the same relative order in the sorted sequence as they were
			in originally.
	\end{description}
}

\part{Answer:}

\begin{description}
	\item[(a)] 
		根據觀察,專注於放入最大的數字 $n$,假設對於先前的排序,總共會有 $n$ 個位置可以插入,
		在插入最大的數字之前,先將所有插入的花費預先支付得到 $n B_{n-1}$。 \\
		接著再考慮其他數字與最大數的之間的移動,在只有 $n-1$ 個數字下,總共有 $(n-1)!$ 種排法,
		每一種排法有 $n$ 種插入方法,插入位置 $i$ 會增加 $i$ 次花費。故得到 $(n-1)! \times \sum_{i=0}^{n-1} 1 = (n-1)! \times \frac{n(n-1)}{2}$
		\begin{align*}
			\left\{\begin{matrix}
				B_0 &=& 0 && \text{if } n = 0\\
				B_n &=& n B_{n-1} + (n-1)! \times \frac{n(n-1)}{2} && \text{if } n > 0
			\end{matrix}\right.
		\end{align*}
		Summation factor $s_n = \frac{a_{n-1} a_n \cdots a_1}{b_n b_{n-1} \cdots b_2} = \frac{1}{n!}$。
		原式改寫
		\begin{align*}
			B_n = n B_{n-1} + (n-1)!  \times \frac{n(n-1)}{2}
			&\Rightarrow \frac{1}{n!}B_n = \frac{1}{(n-1)!} B_{n-1} + \frac{1}{n!}(n-1)! 
				\times \frac{n(n-1)}{2} 
				&& \text{兩邊同除 } s_n\\
			&\Rightarrow S_n = S_{n-1} + \frac{n-1}{2} 
				&& \text{令 } S_n = \frac{1}{n!} B_n\\
			&\Rightarrow S_n = \sum_{i=1}^{n} \frac{i-1}{2} \\
			&\Rightarrow S_n = \frac{1}{4} n(n-1) \\
			&\Rightarrow \frac{1}{n!}B_n = S_n = \frac{1}{4} n(n-1) \\
			&\Rightarrow B_n = \frac{1}{4} n(n-1) \times n!
		\end{align*}
		
		平均運行時間 $$4(E(A_n)) + 6n - 5 = 4(\frac{1}{4} n(n-1) \times n!)/n! + 6n - 5 = n^2 + 5n - 5$$
	\item[(b)]
		不會,因為沒有任何數小於 $-\infty$,故不會移動 $X[\textit{place}]$,因此 $A_n$ 也不會增加。
	\item[(c)]
		這並不是一個 \textit{stable} 排序法,必須修改成 $X[\textit{place}] > \textit{value}$,
		才會成為 \textit{stable} 的插入排序。
\begin{algorithm}
  \caption{Stable Insertion Sort Algorithm}\label{insert}
  \begin{algorithmic}[2]
  	\For{ $i := 2$ \texttt{to} $n$}			\Comment{$n$}
		\State \textit{value} $:= X[i]$		\Comment{$n-1$}
		\State \textit{place} $:= i-1$		\Comment{$n-1$}
		\While{$\textit{place} \ge 1$ \texttt{and} $X[\textit{place}] > \textit{value}$}		\Comment{$A_n + n - 1$}
			\State $X[\textit{place}+1] := X[\textit{place}]$	\Comment{$A_n$}
			\State \textit{place} $:= \textit{place} - 1$	\Comment{$A_n$}
		\EndWhile
		\State $X[\textit{place}+1] := \textit{value}$	\Comment{$n-1$}
	\EndFor
  \end{algorithmic}
\end{algorithm}
\end{description}

\end{document}