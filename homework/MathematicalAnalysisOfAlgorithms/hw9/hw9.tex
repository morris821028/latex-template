%=======================02-713 LaTeX template, following the 15-210 template==================
%
% You don't need to use LaTeX or this template, but you must turn your homework in as
% a typeset PDF somehow.
%
% How to use:
%    1. Update your information in section "A" below
%    2. Write your answers in section "B" below. Precede answers for all 
%       parts of a question with the command "\question{n}{desc}" where n is
%       the question number and "desc" is a short, one-line description of 
%       the problem. There is no need to restate the problem.
%    3. If a question has multiple parts, precede the answer to part x with the
%       command "\part{x}".
%    4. If a problem asks you to design an algorithm, use the commands
%       \algorithm, \correctness, \runtime to precede your discussion of the 
%       description of the algorithm, its correctness, and its running time, respectively.
%    5. You can include graphics by using the command \includegraphics{FILENAME}
%
\documentclass[11pt,fleqn]{article}

\usepackage[UTF8, heading = false, scheme = plain]{ctex} % chinese !!!!
\usepackage{amsmath,amssymb,amsthm}
\usepackage{mathtools}
\usepackage{graphicx}
\usepackage[margin=1in]{geometry}
\usepackage{fancyhdr}
\usepackage{CJK}
\usepackage{manfnt}
\usepackage{listings}
\usepackage{epsdice}

\usepackage{algorithm}% http://ctan.org/pkg/algorithm
\usepackage{algpseudocode}% http://ctan.org/pkg/algorithmicx


\setCJKmainfont{STHeiti} % chinese type

\setlength{\parindent}{0pt}
\setlength{\parskip}{5pt plus 1pt}
\setlength{\headheight}{13.6pt}

% define
\newcommand\question[2]{\vspace{.25in}\hrule\textbf{#1: #2}\vspace{.5em}\hrule\vspace{.10in}}
\renewcommand\part[1]{\vspace{.10in}\textbf{#1}}
\newcommand\correctness{\vspace{.10in}\textbf{Correctness: }}
\newcommand\runtime{\vspace{.10in}\textbf{Running time: }}
\pagestyle{fancyplain}

% header
\lhead{\textbf{\NAME\ (\ANDREWID)}}
\chead{\textbf{HW\HWNUM}}
\rhead{演算法數學分析, \today}


% start content
\begin{document}\raggedright
%Section A==============Change the values below to match your information==================
\newcommand\NAME{Shiang-Yun Yang 楊翔雲}  % your name
\newcommand\ANDREWID{R04922067}     % your andrew id
\newcommand\HWNUM{9}              % the homework number
%Section B==============Put your answers to the questions below here=======================

% no need to restate the problem --- the graders know which problem is which,
% but replacing "The First Problem" with a short phrase will help you remember
% which problem this is when you read over your homeworks to study.

\question{Problem 1} {8-31 An apple is located at vertex $A$ of pentagon $ABCDE$, and
	a worm is located tow vertices away, at $C$. Every day the worm crawls with equals
	probability to one of the two adjacent vertices. Thus after one day the worm is 
	at vertex $B$ or vertex $D$, each with probability $\frac{1}{2}$. After two
	days, the worm might be back at $C$ again, because it has no memory of
	previous positions. When it reaches vertex $A$, it stops to dine.
	\begin{description}
		\item[a] What are the mean and variance of then number of days until dinner?
		\item[b] Let $p$ be the probability that the number of days is 100 or more.
			What does Chebyshev's inequality say about $p$?
		\item[c] What do the tail inequalities (exercise 12) tell us about $p$?
			\begin{align*}
				\text{Pr}(X \ge r) \le x^{-r} P(x) \qquad \text{for } x \ge 1
			\end{align*}
	\end{description}
}

\part{Answer:}

\begin{description}
	\item[a] 假設走到每一個點的生成函數分別為 $A$, $B$, $C$, $D$, $E$,則可列出下式
		\begin{align*}
			\begin{aligned}
				A &= \frac{1}{2}z B + \frac{1}{2}z E\\
				B &= \frac{1}{2}z C\\
				C &= 1 + \frac{1}{2}z B + \frac{1}{2} z D\\
				D &= \frac{1}{2}z C + \frac{1}{2} z E\\
				E &= \frac{1}{2}z D 
			\end{aligned} \qquad
			\Rightarrow \qquad
			\begin{aligned}
				A &= \frac{1}{4}z^2 C + \frac{1}{4}z^2 D\\
				C &= 1 + \frac{1}{4}z^2 C + \frac{1}{2} z D\\
				D &= \frac{1}{2}z C + \frac{1}{4} z^2 D
			\end{aligned} \\
			\qquad
			\Rightarrow \qquad
			\begin{aligned}
				A - \frac{1}{4}z^2 C - \frac{1}{4}z^2 D = 0 && \text{(1)}\\
				\left(\frac{1}{4}z^2-1\right) C + \frac{1}{2} z D = -1 && \text{(2)}\\
				\frac{1}{2}z C + \left(\frac{1}{4} z^2 - 1\right) D = 0 && \text{(3)}
			\end{aligned}
		\end{align*}
		\begin{align*}
			\hline
			\text{(1)} + \text{(2)} + \text{(3)} 
				& \Rightarrow 
					A + \left(\frac{1}{2} z - 1\right) C +
								\left(\frac{1}{2} z - 1\right) D = -1
					&& \text{(4)} \\
			\text{(2)} + \text{3}
				& \Rightarrow 
					\left(\frac{1}{4}z^2 + \frac{1}{2}z - 1\right) C +
						\left(\frac{1}{4}z^2 + \frac{1}{2}z - 1\right) D = -1
					&& \text{(5)} \\
			\text{(5) 代入 (4)}
				& \Rightarrow
					A + \left(\frac{1}{2}-1\right) 
						\frac{-1}{\frac{1}{4}z^2 + \frac{1}{2}z -1} = -1 \\
				& \Rightarrow
					A = \frac{z^2}{4 - 2z - z^2}
		\end{align*}
		\begin{align*}
			\text{Mean(A)} &= {G}'(1) \\
			\text{Var(A)} &= {G}''(1) + {G}'(1) - {G}'(1)^2 \\
			\hline \\
			{G}'(z) &= {\left(\frac{z^2}{4 - 2z - z^2}\right)}' 
					= \frac{8z-2z^2}{(4-2z-z^2)^2} \\
			{G}''(z) &= {\left(\frac{8z-2z^2}{(4-2z-z^2)^2}\right)}' 
					= \frac{(8-4z)(4-2z-z^2)^2 - (8z-2z^2)(-4-4z)}{(4-2z-z^2)^4}
		\end{align*}
		\begin{align*}
			\text{Mean(A)} &= {G}'(1) = 6 \\
			\text{Var(A)} &= {G}''(1) + {G}'(1) - {G}'(1)^2 = 52 + 6 - 36 = 22
		\end{align*}
		
	\item[b] $p$ 為大於等於 100 天才抵達 $A$ 的機率,根據 Chebyshev's inequality
		\begin{align*}
			\text{Pr}(|X - \mu| \ge c \sigma) \le 1 / c^2
		\end{align*}
		\begin{align*}
			\text{Pr}(S \ge 100)
				&= \text{Pr}(|X - 6| \ge 94)
					&& c = 94 / \sqrt{22} \\
				&\le 1 / c^2 = \frac{22}{94^2} \approx 0.0025
		\end{align*}
	\item[c] 代入 exercise 12 最後一條公式
		\begin{align*}
			\text{Pr}(S \ge 100) \le x^{-100} \; \frac{x^2}{4 - 2x - x^2}
				&& \text{for } x \ge 1
		\end{align*}
		找右式的最大值,求一次微分等於零。
		\begin{align*}
			{\left[\frac{1}{x^{98} (4-2x-x^2)}\right]}'
				&= -98 x^{-99} (4-2x-x^2)^{-1}
					+ x^{-98}(-1)(4-2x-x^2)^{-2}(-2-2x) \\
				&= (4-2x-x^2)^{-2} x^{-99} (-98(4-2x-x^2)-x(-2-2x)) \\
				&= (4-2x-x^2)^{-2}(100 x^2 + 198 x - 392)
		\end{align*}
		當 $x = \frac{-99 + \sqrt{49001}}{100}$ 時有最大值,故 $\text{Pr}(S \ge 100) \le 0.00000005$。
\end{description}

\newpage

\question{Problem 2} {9-2 Which function grows faster:
	\begin{description}
		\item[a] $n^{(\ln n)}$ or $(\ln n)^n$ ?
		\item[b] $n^{\ln \ln \ln n}$ or $(\ln n)!$ ?
		\item[c] $(n!)!$ or $((n-1)!)! (n-1)!^{n!}$ ?
		\item[d] $F^2_{\lceil H_n \rceil}$ or $H_{F_n}$ ?
	\end{description}
}

\part{Answer:}

\begin{description}
	\item[a] 同時取 $\log$,分別為 $(\ln n)^2$ 和 $n \ln \ln n$,根據羅畢達法則取極限 
		($n \rightarrow \infty$) 得到 
		$(\ln n)^2 \prec n \ln \ln n \Rightarrow n^{\ln n} \prec (\ln n)^n$。
		故 $(\ln n)^n$ 成長得比較快。
	\item[b] 根據 Stirling's approximation 已知 
		$n! \sim \sqrt{2 \pi n} \left(\frac{n}{e}\right)^n$,
		得到 $(\ln n)! \prec (\ln n)^{\ln n}$,對 $n^{\ln \ln \ln n}$、
		$(\ln n)^{\ln n}$ 同取 $\log$,得 
		\begin{align*}
			\ln n \ln \ln \ln n \prec \ln n \ln \ln n
				\quad &\Rightarrow \quad
			n^{\ln \ln \ln n} \prec (\ln n)^{\ln n} \\
			&\Rightarrow \quad
			n^{\ln \ln \ln n} \prec (\ln n)!
		\end{align*}
		故 $(\ln n)!$ 成長得比較快。
	\item[c] 同時取 $\log$,
		\begin{align*}
			\log ((n!)!) 
				&= \sum\nolimits_{k=0}^{n!} \lg (k!) \\
			\log ((n-1)!)! (n-1)!^{n!} 
				&= \sum\nolimits_{k=0}^{(n-1)!} \lg (k!) + n! \cdot \lg ((n-1)!)
		\end{align*}
		把相同的部分扣除後,得
		\begin{align*}
			\sum\nolimits_{k=(n-1)!+1}^{n!} \lg (k!) \ge n! \cdot \lg ((n-1)!)
		\end{align*}
		所以 $(n!)!$ 成長比較快。
	\item[d] 假設 $\phi = \frac{1 + \sqrt{5}}{2} \approx 1.618$,$F_n \sim \phi^n$ 
		和 $H_n \sim \ln n$。
		\begin{align*}
			F^2_{\lceil H_n \rceil} &
				\asymp \phi^{2 \ln n} = n^{2 \ln \phi} = n^{\ln \phi^2} \\
			H_{F_n} &
				\asymp n \ln \phi
		\end{align*}
		單看 $n$ 指數次方部分,得知 $\ln \phi^2 < 1$,故 $H_{F_n}$ 成長比較快。
\end{description}

\newpage

\question{Problem 3} {9-13 Evaluate $(n + 2 + O(n^{-1}))^n$ with relative 
	error $O(n^{-1})$
}

\part{Answer:}

\begin{align*}
	\left(n + 2 + O(n^{-1})\right)^n 
		&= \left(n(1 + 2n^{-1} + O(n^{-2}))\right)^n \phantom{1\over1}\\
		&= n^n \left(1 + 2n^{-1} + O(n^{-2})\right)^n 
			&& \because e^x = \lim_{n \rightarrow \infty} \left(1 + \frac{x}{n}\right)^n 
			\phantom{1\over1} \\
		&= n^n \exp(n \ln (1 + 2n^{-1} + O(n^{-2}))) \phantom{1\over1} 
			&& \because \text{對 } \ln z \text{ 展開近似公式} \\
		&= n^n \exp(n (2n^{-1} + O(n^{-2}))) \phantom{1\over1} \\
		&= n^n e^{2 + O(n^{-1})} 
			&& \because \text{(9.37)} \quad e^{O\left(f(n)\right)} = 1 + O\left(f(n)\right)
			\phantom{1\over1} \\
		&= n^n \left(e^2 (1 + O(n^{-1}))\right) \phantom{1\over1} \\
		&= n^n e^2 + n^n O\left(n^{-1}\right) \phantom{1\over1} \\
		&= n^n e^2 + O\left(n^{n-1}\right) \phantom{1\over1}
			&& \blacksquare
\end{align*}

\question{Problem 4} {9-14 Prove that
	\begin{align*}
		(n+\alpha)^{n+\beta} 
			= n^{n+\beta} e^{\alpha}\left(1 + \alpha(\beta-\frac{1}{2}\alpha)n^{-1} 
				+ O(n^{-2}) \right).
	\end{align*}
}

\part{Answer:}

\begin{align*}
	(n+\alpha)^{n+\beta} &= \left[n(1+\alpha n^{-1})\right]^{n+\beta} \\
		&= n^{n+\beta} \exp((n+\beta) \ln(1+\frac{\alpha}{n})) \\
		&= n^{n+\beta} \exp((n+\beta) (\frac{\alpha}{n} 
			- \frac{1}{2} \left(\frac{\alpha}{n}\right)^2 + O(n^{-3}))) 
			&& \text{(9.33)}\\
		&= n^{n+\beta} \exp(\alpha-\frac{1}{2}\alpha n^{-1} 
			+ \alpha\beta n^{-1} + O(n^{-2})) \\
		&= n^{n+\beta} \exp(\alpha) \cdot \exp(-\frac{1}{2}\alpha n^{-1}) 
			\cdot \exp(\alpha\beta n^{-1}) \cdot \exp(O(n^{-2})) \\
		&= n^{n+\beta} e^{\alpha} \cdot
			\left(1 - \frac{1}{2}\alpha n^{-1} + O(n^{-2})\right) \cdot
			\left(1 + \alpha\beta n^{-1} + O(n^{-2})\right) \cdot
			\left(1 + O(n^{-2})\right) \\
		&= n^{n+\beta} e^{\alpha} \left(1 + \alpha(\beta-\frac{1}{2}\alpha)n^{-1} 
				+ O(n^{-2}) \right) && \blacksquare
\end{align*}

\question{Problem 5} {9-34 Determinate values $A$ through $F$ such that $(1+1/n)^{n H_n}$ 
	is
	\begin{align*}
		An + B(\ln n)^2 + C \ln n + D + \dfrac{E(\ln n)^2}{n} + \dfrac{F \ln n}{n} + O(n^{-1})
	\end{align*}
}

\part{Answer:}

\begin{align*}
	& \left(1 + \frac{1}{n} \right)^{n H_n} \\
		&= \exp\left(n H_n \cdot \ln\left( 1 + \frac{1}{n} \right) \right) \\
		&= \exp\left(n \cdot 
			\left[ \ln n + \gamma + \frac{1}{2n} - \frac{1}{12n^2} + \frac{1}{120n^4} + O(\frac{1}{n^6}) \right] \cdot 
			\left[ \frac{1}{n} - \frac{1}{2n^2} + \frac{1}{3n^2} - \frac{1}{4n^4} + O(\frac{1}{n^5}) \right]
			\right) \\
		&= \exp\left( \ln n + \gamma + \frac{1}{2n} - \frac{1}{12n^2} 
				- \frac{\gamma}{2n} - \frac{\gamma}{3n^2} - \frac{1}{4n^2}
				- \frac{\ln n}{2n} + \frac{\ln n}{3n^3} + O(n^{-4})\right) \\
		&= n e^r \left(1 - \frac{\ln n}{2n} + \frac{(\ln n)^2}{4n^2\cdot 2!}
					+ \frac{(\ln n)^3}{8n^3\cdot 3!} + O(n^{-4}) \right)
			\left(1 + \frac{\ln n}{3n^2} + \frac{(\ln n)^2}{9n^4\cdot 2!}
					+ \frac{(\ln n)^3}{27n^6\cdot 3!} + O(n^{-6}) \right) \\
		&	\qquad \qquad 
			\left(1 - \frac{\gamma}{2n} + \frac{\gamma^2}{4n^2\cdot 2!} + O(n^{-3}) \right) 
			\left(1 + \frac{\gamma}{3n} + \frac{\gamma^2}{9n^4\cdot 2!} + O(n^{-5}) \right) \\
		&	\qquad \qquad \qquad 
			\left(1 + \frac{1}{2n} + \frac{1}{4n^2 \cdot 2!} + \cdots \right)
			\left(1 - \frac{1}{4n^2} + \frac{1}{16n^4 \cdot 2!} + \cdots \right) 
			\left(1 + O(n^{-4})\right)\\
		&= ne^\gamma + \left(\ln n\right)^2 (0) + 
			\left(\ln n\right) \left(-\frac{1}{2}e^\gamma\right) 
			+ \frac{\left(\ln n\right)^2}{n} \left(\frac{1}{8} e^\gamma\right) \\
		&	\qquad \qquad
			+ \frac{\left(\ln n\right)}{n} 
				\left(\frac{1}{4} e^\gamma \gamma + \frac{1}{3} e^\gamma 
				- \frac{1}{4} e^\gamma\right)
			+ \left(\frac{1}{2} e^\gamma - \gamma e^\gamma \right) + O(n^{-1})
\end{align*}

得 $A = e^\gamma, \; B = 0, \; C = -\frac{1}{2} e^\gamma, \; D = \frac{1}{2} e^\gamma(1-\gamma), \; E = \frac{1}{8} e^\gamma, \; F = \frac{1}{12} e^\gamma (3 \gamma + 1)$。

\newpage

\question{Problem 6} {Let $Y_n$ be the random variable describing the number of the probes
	for unsuccessful search on a binary search tree that contains $n$ keys.
	The generating function for $Y_n$ is defined by
	\begin{align*}
		G_{n}(Z) = \sum_{k} p_{n,k} \; z^k
	\end{align*}
	where $p_{n,k} = \text{Pr}(Y_n = k)$, the probability that $k$ probes are needed to 
	do an unsuccessful search in an $n$-key tree. ...
}

\part{Answer:}

\begin{description}
	\item[(a)] $n$ 個鍵值需要走 $k$ 次走訪的機率 $p_{n,k}$,可以遞迴到 $n-1$ 個鍵值。
		\begin{itemize}
			\item 考慮通過上一次插入的節點,
				相當於在 $n-1$ 個鍵值時,這時候插入第 $n$ 個鍵值 $x_n$,
				若詢問 $y$ 最後落在第 $n$ 個鍵值的節點左右兩側 (最插入一定在葉節點),
				意即 $x_n = y - \frac{1}{2}$ 或 $x_n = y + \frac{1}{2}$,那麼可以得到 
				\begin{align*}
					Y_n(s) = Y_{n-1}(s') + 1.
				\end{align*}
				勢必要多經過一個節點,所以嘗試個數多一。
			\item 考慮不通過上一次插入的節點,相當於在 $n-1$ 個鍵值時,這時候插入第 $n$ 個鍵值 $x_n$,
				若詢問 $y$ 最後落在第 $n$ 個鍵值的節點左右兩側 (最插入一定在葉節點),
				意即 $x_n \neq y - \frac{1}{2}$ 以及 $x_n \neq y + \frac{1}{2}$,那麼可以得到 
				\begin{align*}
					Y_n(s) = Y_{n-1}(s').
				\end{align*}
				不經過這個點,那麼嘗試個數不變。
		\end{itemize}
		利用上述討論的兩種特性,歸納出
		\begin{align*}
			p_{n,k} &= \text{Pr}(Y_{n-1} = k-1 \text{ and } (x_n = y - \dfrac{1}{2} \text{ or } x_n = y + \dfrac{1}{2}) \\
					& + \text{Pr}(Y_{n-1} = k \text{ and } x_n \text{ is one of the other } n-1 \text{ values}) 
		\end{align*}
	\item[(b)] 從 (a) 得到 $G_n(z) = G_{n-1}(z) (2 z + (n-1)) / (n+1)$
		\begin{align*}
			E(Y_n) &= {G}'_n(1) \\
			\text{Var}(Y_n) &= {G}''_n(1) + {G}'_n(1) - {G}'_n(1)^2
		\end{align*}
		已知 $E(Y_1) = 1, \; \text{Var}(Y_1) = 0$
		\begin{align*}
			{G}'_n(z) &= {G}'_{n-1}(z) \frac{2z + (n-1)}{n+1} + G_{n-1}(z) \cdot \frac{2}{n+1} \\
			{G}''_n(z) &= \left[ {G}'_{n-1}(z) \frac{2z + (n-1)}{n+1} + G_{n-1}(z) \cdot \frac{2}{n+1} \right]' \\
				&= {G}''_{n-1}(z)\frac{2z + (n-1)}{n+1} + {G}'_{n-1}(z) \frac{2}{n+1} \\
				& \qquad + {G}'_{n-1}(z)\frac{2z}{n+1}\\
		\end{align*}
		先解期望值,根據遞迴展開
		\begin{align*}
			{G}'_n(1) &= {G}'_{n-1}(1) + \frac{2}{n+1} \\
			\Rightarrow
				E(Y_n) &= E(Y_{n-1}) + \frac{2}{n+1} = 2 H_{n+1} - 2 && \blacksquare
		\end{align*}
		帶入二次微分結果,解出變異數
		\begin{align*}
			\text{Var}(Y_n) &= {G}''_n(1) + {G}'_n(1) - {G}'_n(1)^2 \\
			&= {G}''_{n-1}(1) \frac{2 + (n-1)}{n+1} + {G}'_{n-1}(1) \frac{2}{n+1} \\
			& \qquad + {G}'_{n-1}(1)\frac{2}{n+1} \\
			& \qquad + \left( {G}'_{n-1}(1) + \frac{2}{n+1} \right)\\
			& \qquad - \left( {G}'_{n-1}(1) + \frac{4}{n+1} {G}'_{n-1}(1) + \left(\frac{2}{n+1}\right)^2 \right) \\
			&= {G}''_{n-1}(1) + {G}'_{n-1}(1) - {G}'_{n-1}(1)^2 + \frac{4}{n+1} - \left(\frac{2}{n+1}\right)^2 \\
			&= \text{Var}(Y_{n-1}) + \frac{2}{n+1} - \left(\frac{2}{n+1}\right)^2 \\
			\Rightarrow
				\text{Var}(Y_n) &= 2 H_{n+1} - 4 H^{(2)}_{n+1} + 2 && \blacksquare
		\end{align*}
	\item[(c)] 
		\begin{itemize}
			\item 解 $E[Y_n] = f(n) + c + g(n) + O(1/n)$
			\begin{align*}
				E[Y_n] 
				&= 2 H_{n+1} - 2 \\
				&= 2 \ln (n+1) + 2 \gamma + O(1/n) - 2
			\end{align*}
			得 $f(n) = 2\ln (n+1), \; g(n) = 0, \; c = -2 + 2 \gamma$。
			\item 解 $\text{Var}[Y_n] = f(n) + c + g(n) + O(1/n)$
			\begin{align*}
				\text{Var}[Y_n] 
					&= 2 H_{n+1} - 4 H^{(2)}_{n+1} + 2 \\
					&= \left(2 \ln (n+1) + 2 \gamma + O(1/n)\right)
						- 4 \left( \pi^2 / 6 - O(1/n)\right) + 2 \\
					&= 2 \ln (n+1) + 2 \gamma - 2 \cdot \pi^2 / 3 + 2 + O(1/n)
			\end{align*}
			得 $f(n) = 2\ln (n+1), \; g(n) = 0, \; c = 2 \gamma - 2 \cdot \pi^2 / 3 + 2$。
		\end{itemize}
\end{description}

\end{document}