%=======================02-713 LaTeX template, following the 15-210 template==================
%
% You don't need to use LaTeX or this template, but you must turn your homework in as
% a typeset PDF somehow.
%
% How to use:
%    1. Update your information in section "A" below
%    2. Write your answers in section "B" below. Precede answers for all 
%       parts of a question with the command "\question{n}{desc}" where n is
%       the question number and "desc" is a short, one-line description of 
%       the problem. There is no need to restate the problem.
%    3. If a question has multiple parts, precede the answer to part x with the
%       command "\part{x}".
%    4. If a problem asks you to design an algorithm, use the commands
%       \algorithm, \correctness, \runtime to precede your discussion of the 
%       description of the algorithm, its correctness, and its running time, respectively.
%    5. You can include graphics by using the command \includegraphics{FILENAME}
%
\documentclass[11pt]{article}

\usepackage[UTF8, heading = false, scheme = plain]{ctex} % chinese !!!!
\usepackage{amsmath,amssymb,amsthm}
\usepackage{graphicx}
\usepackage[margin=1in]{geometry}
\usepackage{fancyhdr}
\usepackage{CJK}
\usepackage{manfnt}
\usepackage{listings}

\usepackage{algorithm}% http://ctan.org/pkg/algorithm
\usepackage{algpseudocode}% http://ctan.org/pkg/algorithmicx


\setCJKmainfont{STHeiti} % chinese type

\setlength{\parindent}{0pt}
\setlength{\parskip}{5pt plus 1pt}
\setlength{\headheight}{13.6pt}

% define
\newcommand\question[2]{\vspace{.25in}\hrule\textbf{#1: #2}\vspace{.5em}\hrule\vspace{.10in}}
\renewcommand\part[1]{\vspace{.10in}\textbf{#1}}
\newcommand\correctness{\vspace{.10in}\textbf{Correctness: }}
\newcommand\runtime{\vspace{.10in}\textbf{Running time: }}
\pagestyle{fancyplain}

% header
\lhead{\textbf{\NAME\ (\ANDREWID)}}
\chead{\textbf{HW\HWNUM}}
\rhead{演算法數學分析, \today}


% start content
\begin{document}\raggedright
%Section A==============Change the values below to match your information==================
\newcommand\NAME{Shiang-Yun Yang 楊翔雲}  % your name
\newcommand\ANDREWID{R04922067}     % your andrew id
\newcommand\HWNUM{7}              % the homework number
%Section B==============Put your answers to the questions below here=======================

% no need to restate the problem --- the graders know which problem is which,
% but replacing "The First Problem" with a short phrase will help you remember
% which problem this is when you read over your homeworks to study.

\question{Problem 1} {7-11 This problem, whose three parts are independent, 
	gives practice in the manipulation of generating functions. We assume that 
	$A(z) = \sum_n a_n z^n$, $B(z) = \sum_n b_n z^n$, $C(z) = \sum_n c_n z^n$, 
	and that the coefficients are zero for negative $n$.
	\begin{description}
		\item[a] If $C_n = \sum_{j+2k \le n} a_j b_k$, express $C$ in 
				terms of $A$ and $B$.
		\item[b] If $n b_n = \sum_{k=0}^{n} 2^k a_k / (n-k)!$, express 
				$A$ in terms of $B$.
		\item[c] If $r$ is a real number and if 
				$a_n = \sum_{k=0}^{n} \binom{r+k}{k} b_{n-k}$, express $A$
				in terms of $B$; then use your formula to find coefficients
				$f_k(r)$ such that $b_n = \sum_{k=0}^{n} f_k(r) a_{n-k}$.
	\end{description}
}

\part{Answer:}

\begin{description}
	\item[a]
		目標 $C_n = \sum_{j+2k \le n} a_j b_k$
		\begin{itemize}
			\item $B(z^2) = \sum_n b_n z^{2n}$
			\item $A(z) B(z^2) = \sum_{j,k} (a_j b_k) z^{j+2k}$
			\item $A(z) B(z^2) / (1-z) = \sum_n (\sum_{j+2k \le n} a_j b_k) z^n$
			\item 得 $C(z) = A(z) B(z^2) / (1-z)$
		\end{itemize}
	\item[b]
		目標 $n b_n = \sum_{k=0}^{n} 2^k a_k / (n-k)!$
		\begin{itemize}
			\item 從左式推導,得 $\sum_n n b_n z^n = z B'(z)$
			\item 從右式推導
				\begin{align*}
					A(2z) &= \sum_n 2^n a_n z^n && \text{湊出 } 2^n\\
					A(2z) e^z &= \sum_n 2^n a_n z^n \sum_n \frac{1}{n!} z^n && \text{利用卷積湊出 } 1/(n-k)!\\
					A(2z) e^z &= \sum_n (\sum_{k=0}^n 2^k a_k /(n-k)!) z^n
				\end{align*}
			\item 得 $z B'(z) = A(2z) e^x$
		\end{itemize}
	\item[c]
		當 $a_n = \sum_{k=0}^{n} \binom{r+k}{k} b_{n-k}$ 時
		\begin{itemize}
			\item 左式 $\sum_n a_n z^n = A(z)$
			\item 右式 
				\begin{align*}
					\sum_n ( \sum_{k=0}^{n} \binom{r+k}{k} b_{n-k}) z^n \\
						= \frac{1}{(1-z)^{r+1}} B(z) && \text{Table 335, }
							\sum_{n \ge 0} \binom{c+n-1}{n} z^n \text{對應 }
							\frac{1}{(1-z)^c}
				\end{align*}
			\item 得 $A(z) = B(z) \frac{1}{(1-z)^{r+1}}$
		\end{itemize}
		反求 $b_n = \sum_{k=0}^{n} f_k(r) a_{n-k}$ 的係數 $f_k(r)$
		\begin{itemize}
			\item $B(z) = A(z) (1-z)^{r+1}$
			\item 根據 Table 335 $\sum_{n \ge 0} \binom{c}{n} z^n$ 對應 $(1+z)^c$,$f_k(r)$ 代入 $c \leftarrow r+1, \; z \leftarrow -z$,得 $f_k(r) = \binom{r+1}{k} (-1)^k$。
		\end{itemize}
\end{description}

\question{Problem 2} {7-14 Solve the recurrence
	\begin{align*}
	\begin{matrix}
	g_0 = 0 & \\ 
	g_1 = 1 & \\
 	g_n = -2n g_{n-1} + \sum_{k} \binom{n}{k} g_k g_{n-k} & \text{for } n > 1
	\end{matrix}
	\end{align*}
	by using an exponential generating function.
}

\part{Answer:}

\begin{itemize}
	\item $g_n = -2n g_{n-1} + \sum_{k} \binom{n}{k} g_k g_{n-k} + \delta_{n=1}$
	\item 使用 exponential generating function
		\begin{align*}
			\widehat{G}(z) 
				&= \sum_{n \ge 0} g_n \frac{z^n}{n!} \\
				&= \sum_{n \ge 0} \left [ -2n g_{n-1} + \sum_{k} \binom{n}{k} g_k g_{n-k} + \delta_{n=1} \right ] \frac{z^n}{n!} \\
				&= \sum_{n \ge 0} -2n g_{n-1} \frac{z^n}{n!} + \sum_{n \ge 0} \sum_{k} \left ( \binom{n}{k} g_k g_{n-k} \right ) + z \\
				&= \sum_{n \ge 0} -2 g_{n-1} \frac{z^n}{(n-1)!} + \sum_{n \ge 0} \sum_{k} \left ( \binom{n}{k} g_k g_{n-k} \right ) + z\\
				&= -2 z \widehat{G}(z) + \widehat{G}(z)^2 + z 
				&& \because \text{(7.75), } \widehat{G}(z)^2 = \sum_n \sum_k \binom{n}{k} g_k g_{n-k} z^n
		\end{align*}
	\item 解一元二次次方程式,$\widehat{G}(0)= g_0 = 0$,故正不合。
		\begin{align*}
			\widehat{G}(z) = \frac{1+2z-\sqrt{1+4z^2}}{2}
		\end{align*}
	\item 解一般項 
		\begin{align*}
			\widehat{G}(z) &= \frac{1}{2} + z - \frac{1}{2}(1+4z^2)^{1/2} \\
			& \Rightarrow z^{2n} \frac{g^{2n}}{(2n)!} = - \frac{1}{2} \binom{1/2}{n} 4^n z^{2n} 
				&& \text{Table 335,} \sum_{n \ge 0} \binom{c}{n} z^n \\
		\end{align*}
		化簡 $- \frac{1}{2} \binom{1/2}{n} 4^n$
		\begin{align*}
			- \frac{1}{2} \binom{1/2}{n} 4^n
				&= - \frac{1}{2} \frac{\frac{1}{2}\frac{-1}{2}\cdots\frac{-2n+1}{2}}{n!} 2^{2n} \\
				&= - \frac{\frac{-1}{2}\cdots\frac{-2n+1}{2}}{n!} 2^{2n} \\
				&= (-1)^n \frac{(1 \cdot 3 \cdot 5 \cdots 2n-1) 2^n (1 \cdot 2 \cdot 3 \cdots n-1)}{n!(n-1)!} \\
				&= \frac{1}{n} \binom{2n-2}{n-1} (-1)^n
		\end{align*}
		代回
		\begin{align*}
			z^{2n} \frac{g^{2n}}{(2n)!} = \frac{1}{n} \binom{2n-2}{n-1} (-1)^n z^{2n} \\
			\Rightarrow g^{2n} = \frac{1}{n} \binom{2n-2}{n-1} (-1)^n (2n)!
		\end{align*}
	\item 得一般項
		\begin{align*}
			\left\{\begin{matrix}
				g_1 =& 1 \\
				g_{2n} =& \frac{1}{n} \binom{2n-2}{n-1} (-1)^n (2n)! & \text{for } n > 0 \\
				g_{2n+1} =& 0 & \text{for } n > 0
			\end{matrix}\right.
		\end{align*}
\end{itemize}

\question{Problem 3} {7-15 The \textit{Bell number} $\omega_n$ is the number of ways to
	partition $n$ things into subsets. For example, $\omega_3=5$ because we can partition 
	$\left\{1, 2, 3\right\}$ in the following ways:
	\begin{align*}
	\left\{1, 2, 3\right\}; \left\{1, 2\right\}\cup \left\{3\right\};
	\left\{1, 3\right\}\cup \left\{2\right\};\left\{1\right\}\cup \left\{2,3\right\};
	\left\{1\right\}\cup \left\{2\right\}\cup \left\{3\right\};
	\end{align*}
	\begin{description}
		\item[a]
	Prove that $\omega_{n+1} = \sum_k \binom{n}{k} \omega_{n-k}$, and use this recurrence
	to find a closed form for the exponential generating function 
	$\widehat{P}(z) = \sum_n \omega_n z^n / n!$.
		\item[b]
			Re-evalute $\widehat{P}(z)$ by summing Equation (7,49) on $m$ to see
			if you can get the same result as part(a).
	\end{description}
}

\part{Answer:}

\begin{description}
	\item[a] 證明
		\begin{itemize}
			\item 將第 $n+1$ 個物品與其他 $k$ 個物品放在新的集合,則有 $\binom{n}{k}$ 種,
				剩餘物品的分法有 $\omega_{n-k}$ 種,因此 $\omega_{n+1} = \sum_k \binom{n, k} \omega_{n-k}$。
			\item 給定 $\omega_{n+1} = \sum_k \binom{n}{k} \omega_{n-k}$ 和
					$\widehat{P}(z) = \sum_n \omega_n z^n / n!$
			\item $\widehat{P}(z) = \sum_n \sum_k \binom{n-1}{k} \omega_{n-1-k} z^n / n!$ 
				不好下手,為了使用 (7.75) 的特性,先對 $\widehat{P}(z)$ 微分。
			\item 湊出 7.75 的性質
				\begin{align*}
					\widehat{P}'(z) 
					&= \sum_n \omega_{n} z^{n-1}/(n-1)! \\
					&= \sum_n \sum_k \binom{n-1}{k} \omega_{n-1-k} z^{n-1}/(n-1)! \\
					&= \sum_n \sum_k \binom{n}{k} \omega_{n-k} \frac{1}{n!} z^{n} 
						&& \because \text{(7.75)} \\
					&= e^z \widehat{P}(z)
				\end{align*}
			\item
				微分求得 $\widehat{P}'(z) = e^z \widehat{P}(z)$,解一階線性微分方程 $y' = e^x y$。
				\begin{align*}
					& \frac{y'(x)}{y(x)} = e^x \\
					& \Rightarrow \int \frac{d y(x) / dx}{y(x)} dx = \int e^x dx \\
					& \Rightarrow \int \frac{d y(x)}{y(x)} = e^x + c \\
					& \Rightarrow \ln y = e^x + c\\
					& \Rightarrow y = e^{e^x + c}
				\end{align*}
				得到 $\widehat{P}(z) = e^{e^z+c}$,由於 $\widehat{P}(0)=\omega_0=1$,故 $c=-1$。
			\item 最後,$\widehat{P}(z) = e^{e^z-1}$。
		\end{itemize}
	\item[b]
		\begin{itemize}
			\item $n$ 個物品分在 $m$ 個非空集合的方法數為 $\begin{Bmatrix}n\\m\end{Bmatrix}$,
				故 $\omega_n = \sum_m \begin{Bmatrix}n\\m\end{Bmatrix}$。
			\item 根據 $\omega_n = \sum_m \begin{Bmatrix}n\\m\end{Bmatrix}$
				\begin{align*}
					\widehat{P}(z) &= \sum_{n \ge 0} \sum_m \begin{Bmatrix}n\\m\end{Bmatrix}
						\frac{z^n} / n! \\
					&= \sum_m \sum_{n \ge 0} \begin{Bmatrix}n\\m\end{Bmatrix} z^n / n! \\
					&= \sum_m (e^z-1)^m / m! 
						&& \text{套用 (7.49) 的解}\\
					&= e^{e^z-1}
						&& \text{套用 } e^x = \sum_n x^n / n!
				\end{align*}
			\item 得到與 part a 相同的答案。
		\end{itemize}
\end{description}

\end{document}