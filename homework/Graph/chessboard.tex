\documentclass{article}
\usepackage{tikz}
\usetikzlibrary{arrows}
\usepackage{chessboard} % chessboard drawing
 
\begin{document}
 
\chessboard[tinyboard,vlabelformat=\arabic{filelabel},
  setwhite={Bd5},showmover=false,
  color=red,
  padding=-0.2em,
  pgfstyle=circle,
  markfields={e4,f3,g2,h1,e6,f7,g8,c4,b3,a2,c6,b7,a8,d1,d2,d3,d4,d6,d7,d8,a5,b5,c5,e5,f5,g5,h5}]

  \chessboard[maxfield=c3,vlabelformat=\arabic{filelabel},
  setwhite={Ba2, Ba3, Bb1, Bb3, Bc1, Bc2},
  color=red!75!white,showmover=false,
  pgfstyle=straightmove,
  markmove={a3-a1, b3-b1, c3-c1}
 ]

  \chessboard[maxfield=c3,vlabelformat=\arabic{filelabel},
  setwhite={Ba2, Ba3, Bb1, Bb3, Bc1, Bc2},
  color=red!75!white,showmover=false,
  pgfstyle=straightmove,
  markmove={a1-c1, a2-c2, a3-c3}
 ]
  \chessboard[maxfield=c3,vlabelformat=\arabic{filelabel},
  setwhite={Ba2, Ba3, Bb1, Bb3, Bc1, Bc2},
  color=red!75!white,showmover=false,
  pgfstyle=straightmove,
  markmove={a1-c3, a2-b3, b1-c2}
 ]
 
   \chessboard[maxfield=c3,vlabelformat=\arabic{filelabel},
  setwhite={Ba2, Ba3, Bb1, Bb3, Bc1, Bc2},
  color=red!75!white,showmover=false,
  pgfstyle=straightmove,
  markmove={a3-c1, a2-b1, b3-c2}
 ]
  
\end{document}