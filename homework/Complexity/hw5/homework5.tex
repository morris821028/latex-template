%=======================02-713 LaTeX template, following the 15-210 template==================
%
% You don't need to use LaTeX or this template, but you must turn your homework in as
% a typeset PDF somehow.
%
% How to use:
%    1. Update your information in section "A" below
%    2. Write your answers in section "B" below. Precede answers for all 
%       parts of a question with the command "\question{n}{desc}" where n is
%       the question number and "desc" is a short, one-line description of 
%       the problem. There is no need to restate the problem.
%    3. If a question has multiple parts, precede the answer to part x with the
%       command "\part{x}".
%    4. If a problem asks you to design an algorithm, use the commands
%       \algorithm, \correctness, \runtime to precede your discussion of the 
%       description of the algorithm, its correctness, and its running time, respectively.
%    5. You can include graphics by using the command \includegraphics{FILENAME}
%
\documentclass[11pt]{article}

\usepackage[UTF8, heading = false, scheme = plain]{ctex} % chinese !!!!
\usepackage{amsmath,amssymb,amsthm}
\usepackage{graphicx}
\usepackage[margin=1in]{geometry}
\usepackage{fancyhdr}
\usepackage{CJK}
\usepackage{listings}

\setCJKmainfont{STHeiti} % chinese type

\setlength{\parindent}{0pt}
\setlength{\parskip}{5pt plus 1pt}
\setlength{\headheight}{13.6pt}

% define
\newcommand\question[2]{\vspace{.25in}\hrule\textbf{#1: #2}\vspace{.5em}\hrule\vspace{.10in}}
\renewcommand\part[1]{\vspace{.10in}\textbf{#1}}
\newcommand\algorithm{\vspace{.10in}\textbf{Algorithm: }}
\newcommand\correctness{\vspace{.10in}\textbf{Correctness: }}
\newcommand\runtime{\vspace{.10in}\textbf{Running time: }}
\pagestyle{fancyplain}

% header
\lhead{\textbf{\NAME\ (\ANDREWID)}}
\chead{\textbf{HW\HWNUM}}
\rhead{Theory of Computation, \today}


% start content
\begin{document}\raggedright
%Section A==============Change the values below to match your information==================
\newcommand\NAME{Shiang-Yun Yang 楊翔雲}  % your name
\newcommand\ANDREWID{R04922067}     % your andrew id
\newcommand\HWNUM{5}              % the homework number
%Section B==============Put your answers to the questions below here=======================

% no need to restate the problem --- the graders know which problem is which,
% but replacing "The First Problem" with a short phrase will help you remember
% which problem this is when you read over your homeworks to study.


\question{Problem 1}{Calculate $(2015|999)$ and $(2016|999)$. \textbf{Answers without procedure will get 0 points.}}

Properties of the Jacobi Symbol

\begin{enumerate}
	\item 
		$(ab|m) = (a|m)(b|m)$
	\item 
		$(a|m_1 m_2) = (a|m_1)(a|m_2)$
	\item
		If $a = b \mod m$, then $(a|m) = (b|m)$
	\item
		$(-1|m) = (-1)^{(m-1)/2}$
	\item 
		$(2|m) = (-1)^{(m^2-1)/8}$
	\item
		If $a$ and $m$ are both odd, then $(a|m)(m|a) = (-1)^{(a-1)(m-1)/4}$
\end{enumerate}

\part{Answer:} 

\begin{enumerate}
	\item \begin{equation} \label{eq1}
		\begin{split}
		(2015|999) \; & = (17|999) \\
				   & = (-1)^{(16)(998)/4}(999|17)\\
				   & = (999|17) = (13|17)\\
				   & = (-1)^{(12)(16)/4}(17|13)\\
				   & = (17|13) = (4|13)\\
				   & = (2|13) (2|13) = 1\\
		\end{split}
		\end{equation}
	\item \begin{equation} \label{eq2}
		\begin{split}
		(2016|999) \; &= (18|999) \\
				   &= (2|999)(9|999)\\
				   &= (-1)^{(999^2-1)/8}(9|999) \\
				   &= (-1)^{124750}(9|999) = (9|999) \\
				   &= (-1)^{(8)(998)/4}(999|9) = (999|9)\\
				   &= (0|9) = 0
		\end{split}
		\end{equation}
\end{enumerate}

% \part{b}

\question{Problem 2}{Let the two primes $p=41$ and $q =17$ be given for RSA. Which of the parameters $e_1=32$, $e_2=49$ is a valid RSA exponent? What is the value of the private key $d$?}

\part{Answer:}

\begin{enumerate}
	\item 
		Compute $n = p \times q = 41 \times 17 = 697$
	\item 
		Compute the totient of the product as $\varphi(n) = (p-1)(q-1) = 40 \times 16 = 640$
	\item 
		Choose any number $ 1 < e < \varphi(n)$ that is coprime to $\varphi(n)$. The $e_1=32$ is not a valid RSA exponent because greatest common divisor $\mathit{gcd}(32, 640) \neq 1$, but $e_2=49$ is valid because $\mathit{gcd}(49, 640) = 1$.
	\item
		Compute $d$, the modular multiplicative inverse of $e \mod \varphi(n)$. \\
		We get $d=209$. \\
		$e \times d \mod \varphi(n) = 49 \times 209 \mod 640 = 10241 \mod 640 = 1$, using Euclidean algorithm following steps:
		\begin{equation} \label{eq3}
		\begin{split}
			640 \; & = 49 \times 13 + 3, \; (640, 49) = (49, 3) \\
			49  \; & = 3 \times 16 + 1, \; (49, 3) = (3, 1) \\
			3   \; & = 1 \times 3, \; (3, 1) = 1 
		\end{split}
		\end{equation}
		Assume $a = 640, \; b = 49$
		\begin{equation} \label{eq3}
		\begin{split}
			a \; & = b \times 13 + (a - 13 b)\\
			b \; & = (a - 13 b) \times 16 + (-16 a - 209 b)\\
			(a - 13 b) \; & = (-16 a - 209 b) \times 3
		\end{split}
		\end{equation}
		then, $-16a - 209b = 1 \Rightarrow 16 \times 640 + 209 \times 49 = 1 \Rightarrow 209 \times 49 = 1\mod 640$
	\item
		$e_2 = 49$ is a valid RSA exponent, and private key $d = 209$ according by $e_2$.
\end{enumerate}

\end{document}
