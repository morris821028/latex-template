%=======================02-713 LaTeX template, following the 15-210 template==================
%
% You don't need to use LaTeX or this template, but you must turn your homework in as
% a typeset PDF somehow.
%
% How to use:
%    1. Update your information in section "A" below
%    2. Write your answers in section "B" below. Precede answers for all 
%       parts of a question with the command "\question{n}{desc}" where n is
%       the question number and "desc" is a short, one-line description of 
%       the problem. There is no need to restate the problem.
%    3. If a question has multiple parts, precede the answer to part x with the
%       command "\part{x}".
%    4. If a problem asks you to design an algorithm, use the commands
%       \algorithm, \correctness, \runtime to precede your discussion of the 
%       description of the algorithm, its correctness, and its running time, respectively.
%    5. You can include graphics by using the command \includegraphics{FILENAME}
%
\documentclass[11pt]{article}

\usepackage[UTF8, heading = false, scheme = plain]{ctex} % chinese !!!!
\usepackage{amsmath,amssymb,amsthm}
\usepackage{graphicx}
\usepackage[margin=1in]{geometry}
\usepackage{fancyhdr}
\usepackage{CJK}
\usepackage{listings}

\setCJKmainfont{STHeiti} % chinese type

\setlength{\parindent}{0pt}
\setlength{\parskip}{5pt plus 1pt}
\setlength{\headheight}{13.6pt}

% define
\newcommand\question[2]{\vspace{.25in}\hrule\textbf{#1: #2}\vspace{.5em}\hrule\vspace{.10in}}
\renewcommand\part[1]{\vspace{.10in}\textbf{#1}}
\newcommand\algorithm{\vspace{.10in}\textbf{Algorithm: }}
\newcommand\correctness{\vspace{.10in}\textbf{Correctness: }}
\newcommand\runtime{\vspace{.10in}\textbf{Running time: }}
\pagestyle{fancyplain}

% header
\lhead{\textbf{\NAME\ (\ANDREWID)}}
\chead{\textbf{HW\HWNUM}}
\rhead{Theory of Computation, \today}


% start content
\begin{document}\raggedright
%Section A==============Change the values below to match your information==================
\newcommand\NAME{Shiang-Yun Yang 楊翔雲}  % your name
\newcommand\ANDREWID{R04922067}     % your andrew id
\newcommand\HWNUM{4}              % the homework number
%Section B==============Put your answers to the questions below here=======================

% no need to restate the problem --- the graders know which problem is which,
% but replacing "The First Problem" with a short phrase will help you remember
% which problem this is when you read over your homeworks to study.


\question{Problem 1}{Find all the primitive roots of 5 and all the primitive roots of 7.}

\part{Answer:} 

\textbf{Theorem 52:} \textit{A number $p > 1$ is a prime if and only if there is a number $1 < r < p$ such that}

\begin{itemize}
	\item	$r^{p-1} = 1 \mod p$, and
	\item	$r^{(p-1)/q} \neq 1 \mod p$ \textit{ for all prime divisors $q$ of $p-1$}
\end{itemize}

\begin{enumerate}
	\item the primitive roots of 5: 
		$p = 5$, $p-1 = 2^2$, $q \in \left \{ 2 \right \}$
		\begin{itemize}
			\item test 2 : $2^{(5-1)/2} = 2^2 = 4 \not\equiv 1 \mod 5$, so 2 is a primitive root of 5.
			\item test 3 : $3^{(5-1)/2} = 3^2 = 9 \not\equiv 1 \mod 5$, so 3 is a primitive root of 5.
			\item test 4 : $4^{(5-1)/2} = 4^2 = 16 \equiv 1 \mod 5$, so 4 is \textbf{not} a primitive root of 5.
		\end{itemize}
		
		Finally, we get 2 and 3 are the primitive root of 5.
	\item the primitive roots of 7:
		$p = 7$, $p-1 = 2 \times 3$, $q \in \left \{ 2, \; 3 \right \}$
		\begin{itemize}
			\item test 2 : \\
			$2^{(7-1)/2} = 2^3 = 8 \equiv 1 \mod 7$, so 2 is \textbf{not} a primitive root of 7.
			\item test 3 : \\
			$3^{(7-1)/2} = 3^3 = 27 \not\equiv 1 \mod 7$ and $3^{(7-1)/3} = 3^2 = 9 \not\equiv 1 \mod 7$, so 3 is a primitive root of 7.
			\item test 4 : \\
			$4^{(7-1)/2} = 4^3 = 64 \equiv 1 \mod 7$, so 4 is \textbf{not} a primitive root of 7.
			\item test 5 : \\
			$5^{(7-1)/2} = 5^3 = 125 \not\equiv 1 \mod 7$ and $5^{(7-1)/3} = 5^2 = 25 \not\equiv 1 \mod 7$, so 5 is a primitive root of 7.
			\item test 6 : \\
			$6^{(7-1)/2} = 6^3 = 216 \not\equiv 1 \mod 7$ and $6^{(7-1)/3} = 6^2 = 35 \equiv 1 \mod 7$, so 6 is \textbf{not} a primitive root of 7.
		\end{itemize}
		Finally, we get 3 and 5 are the primitive root of 7.
\end{enumerate}


% \correctness This is an argument  that this algorithm returns the correct answer.

% \runtime Describe here, in big-Oh, the running time and your reasoning for it.

% \part{b}

\question{Problem 2}{We know that $3\textsc{-SAT}$ is NP-complete. Show that for $n > 3$, $n\textsc{-SAT}$ is also NP-complete. (You don't need to show that is in NP.)}

\part{Answer:}

\begin{enumerate}
	\item 首先,我們先證明 $4\textsc{-SAT}$ 是 NP-complete。接著推論當 $n\textsc{-SAT}$ 是 NP-complete,則 $(n+1)\textsc{-SAT}$ 是 NP-complete。
	\item 由於 $4\textsc{-SAT}$ 存在 nondeterministic polynomial-time algorithm,則 $4\textsc{-SAT} \in \textsc{NP}$ 
	\item 為了證明 $4\textsc{-SAT}$ 屬於 NP-hard,將 $3\textsc{-SAT}$ reduce to $4\textsc{-SAT}$。如果 $4\textsc{-SAT}$ 能夠找到一種 truth assignment 滿足,則 $3\textsc{-SAT}$ 也存在一種 truth assignment 滿足。
	\item 用 $\phi$ 表示 $3\textsc{-SAT}$ 的格式,轉換成 ${\phi}'$ 表示 $4\textsc{-SAT}$ 的格式。對每一個 clause $(x \vee y \vee z) \in \phi$ 轉換成 clause $(x \vee y \vee z \vee h) \wedge (x \vee y \vee z \vee \lnot h) \in {\phi}'$,其中 $h$ 是額外新增的變數,明顯地轉換只需要多項式時間即可完成。
	\item 最後,推論 \textit{if $\phi$ is satisfiable if and only if $R(\phi) = {\phi}'$ is satisfiable.}
		\begin{itemize}
			\item
			 如果 $(x \vee y \vee z) \in \phi$ 滿足, 則 $(x \vee y \vee z \vee h) \wedge (x \vee y \vee z \vee \lnot h) \in {\phi}'$ 按照 $\phi$ 的 truth assignment 不管 $h$ 的值為何都會滿足。
			 \item
			 如果 $(x \vee y \vee z \vee h) \wedge (x \vee y \vee z \vee \lnot h) \in {\phi}'$ 滿足, 則 $(x \vee y \vee z) \in \phi$ 按照 ${\phi}'$ 的 truth assignment 也會滿足。因為 $h$ 和 $\lnot h$ 為相反的布林值,則 $(x \vee y \vee z) \in \phi$ 為真。
		\end{itemize}
	\item
		證明 $n\textsc{-SAT}$ is NP-complete for $n \ge 3$。
		\begin{itemize}
			\item $n = 3$, $n = 4$ 成立。
			\item 假設 $n = m$ 時滿足 $m\textsc{-SAT}$ 為 NP-complete,則 $n = m+1$ 時也滿足。證明如下: \\
				將 $m\textsc{-SAT}$ reduce $(m+1)\textsc{-SAT}$。若 $\phi \in m\textsc{-SAT}$,則 $R(\phi) = {\phi}'$ 如下: \\
				$$R: \text{clause } \bigvee x_i \in \phi \Rightarrow \text{clause } \left ( \left ( \bigvee x_i \right ) \vee h \right ) \wedge \left ( \left ( \bigvee x_i \right ) \vee \lnot h \right ) \in {\phi}'$$ \\
				
				其中 $h$ 是額外新增的變數,同 5. 的說明,得到 $m\textsc{-SAT}$ reduce to $(m+1)\textsc{-SAT}$,$(m+1)\textsc{-SAT}$ 是 NP-hard 也是 NP,故 $(m+1)\textsc{-SAT}$ 是 NP-complete。
			\item 由數學歸納法得證 $n\textsc{-SAT}$ is NP-complete for $n \ge 3$。
		\end{itemize}
\end{enumerate}

\end{document}
