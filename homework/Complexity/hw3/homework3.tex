%=======================02-713 LaTeX template, following the 15-210 template==================
%
% You don't need to use LaTeX or this template, but you must turn your homework in as
% a typeset PDF somehow.
%
% How to use:
%    1. Update your information in section "A" below
%    2. Write your answers in section "B" below. Precede answers for all 
%       parts of a question with the command "\question{n}{desc}" where n is
%       the question number and "desc" is a short, one-line description of 
%       the problem. There is no need to restate the problem.
%    3. If a question has multiple parts, precede the answer to part x with the
%       command "\part{x}".
%    4. If a problem asks you to design an algorithm, use the commands
%       \algorithm, \correctness, \runtime to precede your discussion of the 
%       description of the algorithm, its correctness, and its running time, respectively.
%    5. You can include graphics by using the command \includegraphics{FILENAME}
%
\documentclass[11pt]{article}

\usepackage[UTF8, heading = false, scheme = plain]{ctex} % chinese !!!!
\usepackage{amsmath,amssymb,amsthm}
\usepackage{graphicx}
\usepackage[margin=1in]{geometry}
\usepackage{fancyhdr}
\usepackage{CJK}
\usepackage{listings}

\setCJKmainfont{STHeiti} % chinese type

\setlength{\parindent}{0pt}
\setlength{\parskip}{5pt plus 1pt}
\setlength{\headheight}{13.6pt}

% define
\newcommand\question[2]{\vspace{.25in}\hrule\textbf{#1: #2}\vspace{.5em}\hrule\vspace{.10in}}
\renewcommand\part[1]{\vspace{.10in}\textbf{#1}}
\newcommand\algorithm{\vspace{.10in}\textbf{Algorithm: }}
\newcommand\correctness{\vspace{.10in}\textbf{Correctness: }}
\newcommand\runtime{\vspace{.10in}\textbf{Running time: }}
\pagestyle{fancyplain}

% header
\lhead{\textbf{\NAME\ (\ANDREWID)}}
\chead{\textbf{HW\HWNUM}}
\rhead{Theory of Computation, \today}


% start content
\begin{document}\raggedright
%Section A==============Change the values below to match your information==================
\newcommand\NAME{Shiang-Yun Yang 楊翔雲}  % your name
\newcommand\ANDREWID{R04922067}     % your andrew id
\newcommand\HWNUM{3}              % the homework number
%Section B==============Put your answers to the questions below here=======================

% no need to restate the problem --- the graders know which problem is which,
% but replacing "The First Problem" with a short phrase will help you remember
% which problem this is when you read over your homeworks to study.


\question{Problem 1}{Denote $L(M)$ as the language $L$ accepted by the Turing machine (TM) $M$. Determine if the following languages are decidable and explain.}

\begin{enumerate}
	\item $L_1 = \{ M | M \text{ is a TM and there exists and input on which $M$ halts with in } |M| \text{ steps.} \}$
	\item $L_2 = \{ M | M \text{ is a TM and } L(M) \text{ is uncountable.} \}$
\end{enumerate}


\part{Answer:} % \algorithm Describe algorithm here

\begin{enumerate}
	\item
		\begin{enumerate}
			\item 假設 TM $T$ decides $L_1$,且長度 $n = |M|$
			\item 照字典順序窮舉所有字串長度介於 $[ 1, n ]$ 之間,最多有 $O(|\Sigma|^n)$ 個字串,因此字串的集合大小是有限的。
			\item TM $T$ 一定會停止在 accept 或 reject 的狀態,如果是 accept 必然會在 $n$ 次以內停止。否則會在模擬 $n$ 步之後回答 reject。
			\item 由於存在一個 TM $T$ decides $L_1$,因此 $L_1$ is decidable。
		\end{enumerate}
	\item
		\begin{enumerate}
			\item $L_2$ 是一個空集合,所有 Turing Machine $M$ 對應的 $L(M)$ is countable,因為 $\Sigma^{\ast}$ is countable,不存在任何一台 TM 對應的 $L(M)$ is uncountable。
			\item 假設 TM $T$ decides $L_2$,則對於任何輸入 $M$ 都會 reject。
			\item \begin{lstlisting}
D(M) {
	return 'reject';
}
\end{lstlisting}
			\item 由於存在一個 TM $T$ decides $L_2$,因此 $L_2$ is decidable。
		\end{enumerate}
\end{enumerate}


% \correctness This is an argument  that this algorithm returns the correct answer.

% \runtime Describe here, in big-Oh, the running time and your reasoning for it.

% \part{b}

\question{Problem 2}{Given $L_3 = \{ M ; x ; y \; | \; M(x) = y \}$ where $M$ is Turing machine (TM), $x$ and $y$ are strings. Use reduction to show that $L_3$ is undecidable. (You are not allowed to cite Rice's theorem.)}

\part{Answer:}

\begin{lstlisting}
                     Halting Problem
        +-------------------------------------------+
        |   M; x; yes  +-------+                    |
        |  +---------->+ DL3   +--+  +--------+     |
 M; x   |  |           +-------+  +->+        +-----> Yes
+-----> +--+                         | OR GATE|     |
        |  |           +-------+  +->+        +-----> No
        |  +---------->+ DL3   +--+  +--------+     |
        |   M; y; no   +-------+                    |
        +-------------------------------------------+

\end{lstlisting}

\begin{itemize}
	\item Halting Problem $H(M; x)$ reduces to $L_3$ via a transform $R$.
	\item Let $D_{L_3}$ decides $L_3$.
	\item Transform $R$ is defined by
	$$H(M; x) = \{ M; x | D_{L_3}(M; x; \text{ yes}) = \text{ yes} \text{ or } D_{L_3}(M; x; \text{ no}) = \text{ yes}\}$$
	\item If $L_3$ is decidable, then $H(M; x)$ is also decidable. 
	\item Due to $H$ is undecidable, $L_3$ is undecidable.
\end{itemize}

\end{document}
